%%%%%%%%%%%%%%%%%%%%%%%%%%%%%%%%%%%%%%%%%%
% Engineering problems / LaTeX Template
%		Semester 5
%		Institut d'Optique Graduate School
%%%%%%%%%%%%%%%%%%%%%%%%%%%%%%%%%%%%%%%%%%
%	5N-OpE	/ Découverte du C++
%%%%%%%%%%%%%%%%%%%%%%%%%%%%%%%%%%%%%%%%%%
%
% Created by:
%	Julien VILLEMEJANE - 18/jun/2024
% Modified by:
%	
%
%%%%%%%%%%%%%%%%%%%%%%%%%%%%%%%%%%%%%%%%%%
% Professional Newsletter Template
% LaTeX Template
% Version 1.0 (09/03/14)
%
% Created by:
% Bob Kerstetter (https://www.tug.org/texshowcase/) and extensively modified by:
% Vel (vel@latextemplates.com)
% 
% This template has been downloaded from:
% http://www.LaTeXTemplates.com
%
% License:
% CC BY-NC-SA 3.0 (http://creativecommons.org/licenses/by-nc-sa/3.0/)
%
%%%%%%%%%%%%%%%%%%%%%%%%%%%%%%%%%%%%%%%%%

\documentclass[10pt]{article} % The default font size is 10pt; 11pt and 12pt are alternatives

\input{../../../_assets/latex/5N_ONIP_structure.tex} % Include the document which specifies all packages and structural customizations for this template
\usepackage{amsmath}
\usepackage{listings}
\lstdefinestyle{cppStyle}{
    language=C++,
    basicstyle=\small\ttfamily,
    keywordstyle=\color{blue}\bfseries,
    commentstyle=\color{green!60!black},
    stringstyle=\color{red},
    showstringspaces=false,
    breaklines=true,
    frame=single,
    numbers=left,
    numberstyle=\tiny\color{gray},
    captionpos=b
}



%----------------------------------------------------------------------------------------
%	DOCUMENT INFORMATIONS
%----------------------------------------------------------------------------------------
\def\module{Découverte du C++}
\def\submodule{C++}
\def\moduleSmall{C++}
\def\year{2024-2025}
\def\problem{Introduction à C++}
\def\problemName{Maitriser les concepts de base du langage C++}

\def\validation{}

\def\scheduleCM{0}
\def\scheduleTD{2}
\def\scheduleTDcomputer{0}
\def\scheduleTP{0}

\def\workingTeam{Seul ou par binôme}

\def\workingSpecial{}

\def\keywords{Typage; Fonction; Tableaux}


\begin{document}
%----------------------------------------------------------------------------------------
%	HEADER IMAGE
%----------------------------------------------------------------------------------------

\begin{figure}[H]
\centering\includegraphics[width=0.5\linewidth]{../../../_assets/latex/logo_iogs.png}
\end{figure}

%----------------------------------------------------------------------------------------
%	SIDEBAR - FIRST PAGE
%----------------------------------------------------------------------------------------

\begin{minipage}[t]{.33\linewidth} % Mini page taking up 30% of the actual page
\begin{mdframed}[style=sidebar,frametitle={\module}] % Sidebar box

%-----------------------------------------------------------
%	DOCUMENT DESCRIPTION
\begin{center}

\textit{\large \centering \year}
\end{center}


\centerline {\rule{.70\linewidth}{.25pt}} % Horizontal line

\begin{center}
	\textit{\large \moduleSmall}
\end{center}

\centerline {\rule{.70\linewidth}{.25pt}} % Horizontal line

\begin{center}
	\textbf{\problem} ( \validation )
\end{center}

\centerline {\rule{.70\linewidth}{.25pt}} % Horizontal line

%-----------------------------------------------------------

\textbf{Concepts étudiés}

\begin{itemize}
\item[\textsc{\scriptsize [Num]}] Typage des données
\item[\textsc{\scriptsize [Num]}] Fonctions et tableaux
\end{itemize}

\centerline {\rule{.70\linewidth}{.25pt}} % Horizontal line

%-----------------------------------------------------------

\textbf{Mots clefs}

\keywords

\centerline {\rule{.70\linewidth}{.25pt}} % Horizontal line

%-----------------------------------------------------------

\textbf{Sessions}

\begin{itemize}
\item[\textbf{\scheduleCM}] Cours(s) - 1h30
\item[\textbf{\scheduleTD}] TD(s) - 1h30
\item[\textbf{\scheduleTDcomputer}] TD(s) Machine - 2h00
\item[\textbf{\scheduleTP}] TP(s) - 4h30
\end{itemize}

\centerline {\rule{.70\linewidth}{.25pt}} % Horizontal line

{\large Travail}

\textbf{\workingTeam}

\textbf{\workingSpecial}

%-----------------------------------------------------------
\end{mdframed}


\begin{minipage}[t]{.95\linewidth}
\textbf{Institut d'Optique}\\
Graduate School, \textit{France}\\
\href{https://www.institutoptique.fr}{https://www.institutoptique.fr}

\medskip
\textbf{GitHub - Digital Methods}

\href{https://github.com/IOGS-Digital-Methods}{https://github.com/IOGS-Digital-Methods}

\end{minipage}

\end{minipage}\hfill % End the sidebar mini page 
% NO SPACE BETWEEN THE END OF SIDEBAR AND BEGIN OF MAIN PART
%----------------------------------------------------------------------------------------
%	MAIN BODY - FIRST PAGE
%----------------------------------------------------------------------------------------
%
\begin{minipage}[t]{.65\linewidth} % Mini page taking up 65% of the actual page

\hypertarget{context}{\heading{\huge \problemName}{6pt}} % \hypertarget provides a label to reference using \hyperlink{label}{link text}

\centerline {\rule{.70\linewidth}{.25pt}} % Horizontal line

%% Short introduction 

Malgré une forte croissance de l'utilisation du langage \textbf{Python} dans le monde scientifique (analyse de données, machine learning, automatisation, prototypage rapide...), le langage \textbf{C++} reste encore parmi les plus performants en terme de rapidité d'exécution.

C'est notamment grâce à son étape de \textbf{compilation} \textbf{langage de programmation compilé} et sa compatibilité avec le langage C qui en font un des langages de programmation les plus utilisés dans les applications où la performance est critique (logiciels systèmes - exploitation et drivers, traitement de données, systèmes embarqués).

\centerline {\rule{.70\linewidth}{.25pt}} % Horizontal line

L'objectif de ces deux séances est de prendre en main un environnement de développement simple (CodeBlocks) et de découvrir les bases du langage C++, à travers des exemples à tester et à modifier.

Alors, à vos claviers...


%%

\bigskip

%----------------------------------------------------------------------------------------
%	IN-TEXT BOX / Intended learning outcomes
%----------------------------------------------------------------------------------------

\begin{mdframed}[style=aavbox,frametitle={Acquis d'Apprentissage Visés}]

En résolvant ces problèmes, les étudiant$\cdot$e$\cdot$s seront capables de  :

\centerline {\rule{.40\linewidth}{.1pt}} % Horizontal line

\begin{enumerate}
\item \textbf{Compiler un code en langage C++}
\item \textbf{Définir des variables}
\item \textbf{Définir des fonctions} pour faciliter la réutilisabilité d'un code
\item \textbf{Définir et utiliser des tableaux}
\item \textbf{Utiliser une bibliothèque de fonctions}
\end{enumerate}

\end{mdframed}


\medskip
%-----------------------------------------------------------

\hypertarget{ressources}{\heading{Ressources}{6pt}} % \hypertarget provides a label to reference using \hyperlink{label}{link text}

Cette séquence est basée sur le langage C++.

Nous utiliserons l'environnement \textbf{CodeBlocks} (gratuit).

Des tutoriels C++ sont en cours de développement à l'adresse suivante : \href{https://iogs-lense-training.github.io/cpp-basics-oop/}{https://iogs-lense-training.github.io/cpp-basics-oop/}. 

\end{minipage} % End the main body - first page mini page

%----------------------------------------------------------------------------------------
%	MAIN BODY - SECOND PAGE
%----------------------------------------------------------------------------------------
\newpage

%-----------------------------------------------------------

\hypertarget{stepbystep}{\heading{Travail à réaliser}{6pt}} % \hypertarget provides a label to reference using \hyperlink{label}{link text}

\textbf{Il est suggéré de créer un nouveau projet à chaque exercice afin de conserver une trace des programmes.}

%-----------------------------------
\centerline {\rule{.70\linewidth}{.25pt}} % Horizontal line
\begin{center}
\textsc{\large Séance 1}
\end{center}
\centerline {\rule{.70\linewidth}{.25pt}} % Horizontal line
%-----------------------------------



%-----------------------------------
\textbf{Exercice 1 / Création d'un projet sous CodeBlocks (HelloWorld)}

On se propose dans un premier temps de créer un nouveau projet C++ avec l'interface de développement CodeBlocks.

\begin{enumerate}
	\item Lancer l'application CodeBlocks.
	\item Puis sélectionner \textsc{File/New/Project...} (\textsc{Fichier/Nouveau/Projet...} en français).
	\item Sélectionner ensuite le modèle (\textit{template}) \textsc{Console Application}.
	\item Sélectionner le langage \textsc{C++}.
	\item Donner un titre à votre projet dans le champ \textsc{Project Title} puis sélectionner le répertoire dans lequel vous souhaitez sauvegarder votre projet dans le champ \textsc{Folder to create project in}.
	\item Dans la fenêtre suivante, sélectionner le compilateur (\textit{Compiler}) : \textsc{GNU GCC Compiler} (installé avec CodeBlocks en sélectionnant la version incluant MingW).
	\item Après validation, votre projet est créé dans la fenêtre principale de l'application. Ouvrir le dossier \textsc{Sources} dans le manager de projet et ouvrir le fichier intitilé \textsc{main.cpp}
\end{enumerate}

Il faut ensuite compiler le projet (\textit{Build}). Pour cela, il existe le raccourci clavier \textsl{CTRL + F9}.

Une fois la compilation réussie avec succès, il reste à exécuter le programme (\textit{Run}). Pour cela, il existe le raccourci clavier \textsl{CTRL + F10}.


\centerline {\rule{.50\linewidth}{.25pt}} % Horizontal line
%-----------------------------------
\textbf{Exercice 2 / Structure de base d'un code C++ et interaction}

On se propose d'étudier le code \textit{\textbf{exo1\_2.cpp}} fourni.

\begin{enumerate}
	\item Créez un nouveau projet avec CodeBlocks.
	\item Copiez-collez le code dans le fichier main.cpp du projet.
	\item Exécutez le programme. Expliquez le code.
	\item Modifiez le code pour demander à l'utilisateur de saisir le valeur de la variable \textit{resistance} avant de l'afficher.
\end{enumerate}


\centerline {\rule{.50\linewidth}{.25pt}} % Horizontal line
%-----------------------------------
\textbf{Exercice 3 / Mauvais calculs ?}

On se propose d'étudier le code \textit{\textbf{exo1\_3.cpp}} fourni.

\begin{enumerate}
	\item Créez un nouveau projet avec CodeBlocks.
	\item Copiez-collez le code dans le fichier main.cpp du projet.
	\item Exécutez le programme. Est-ce le bon résultat ?
	\item Expliquez l'erreur et modifiez le programme pour qu'il affiche le bon résultat.
	\item Réécrivez les lignes contenant des \textit{printf} à l'aide de l'opérateur de sortie du C++ (\textit{cout}).
	\item Ajoutez la possibilité à l'utilisateur de saisir les valeurs des variables \textbf{a} et \textit{b}.
\end{enumerate}


\centerline {\rule{.50\linewidth}{.25pt}} % Horizontal line
\newpage
%-----------------------------------
\textbf{Exercice 4 / Conditions et itérations}

Le code suivant permet de \textbf{faire un test} sur la variable a et d'afficher si sa valeur est supérieure à 8 ou non.

\begin{lstlisting}[style=cppStyle]
int a = 10;

if(a > 8){
	printf("a > 8\n");
}
else{
	printf("a <= 8\n");
}
\end{lstlisting}


Le code suivant permet d'\textbf{itérer un nombre de fois prédéfini} des actions (ici l'affichage des 5 premiers nombres entiers impairs).

\begin{lstlisting}[style=cppStyle]
int i;

for(i = 0; i < 5; i++){
	printf("[%d] -> %d \n", i, i*2+1);
}
\end{lstlisting}

On propose d'étudier à présent les deux codes suivants :

\begin{lstlisting}[style=cppStyle]
int i = 0;

do{
	printf("[%d] -> %d \n", i, i*2+1);
	i++;
}while(i < 5);
\end{lstlisting}

\begin{lstlisting}[style=cppStyle]
int i = 0;

while(i < 5){
	printf("[%d] -> %d \n", i, i*2+1);
	i++;
};
\end{lstlisting}

\begin{enumerate}
	\item A quoi servent ces deux codes ? Quelle est leur différence ?
	\item Écrire un programme qui demande à l'utilisateur de taper un entier N entre 0 et 20 bornes incluses et continue de demander tant que la valeur n'est pas correcte. La valeur finale sera affichée.
\end{enumerate}


\centerline {\rule{.50\linewidth}{.25pt}} % Horizontal line
%-----------------------------------
\textbf{Exercice 5 / Fonctions}

On se propose d'étudier le code \textit{\textbf{exo1\_5.cpp}} fourni, qui contient une fonction \textit{somme} et un programme principal qui appelle cette fonction avec des paramètres particuliers.

\begin{lstlisting}[style=cppStyle]
/* Definition de la fonction somme */
int somme(int a, int b){
	int s = a + b;
	return s;
}
\end{lstlisting}


\begin{enumerate}
	\item Écrire une fonction qui calcule u(N) défini par :

\qquad \qquad u(n+1)=3*u(n)+4

	\item Ecrire un programme de test qui demande à l'utilisateur de saisir un nombre entre 0 et 20 et qui exécute la fonction écrite précédemment.	
\end{enumerate}






%----------------------------------------------------------------------------------------
\newpage
%-----------------------------------------------------------

\hypertarget{stepbystep}{\heading{Travail à réaliser (suite)}{6pt}} % \hypertarget provides a label to reference using \hyperlink{label}{link text}

\centerline {\rule{.70\linewidth}{.25pt}} % Horizontal line
\begin{center}
\textsc{\large Séance 2}
\end{center}
\centerline {\rule{.70\linewidth}{.25pt}} % Horizontal line
%-----------------------------------

\textbf{Exercice 1 / Bibliothèque de fonctions - tab1D}

Soit les fichiers \textit{\textbf{tab1D.h}} et \textit{\textbf{tab1D.cpp}} contenant des fonctions associées à la gestion de tableaux en C++ et le fichier \textit{\textbf{exo2\_1.cpp}} contenant un test des fonctions contenues dans les fichiers précédemment cités.

\begin{enumerate}
	\item Créez un nouveau projet avec CodeBlocks.
	\item Copiez-collez le code du fichier \textit{\textbf{exo2\_1.cpp}} fourni dans le fichier main.cpp du projet.
	\item Ajoutez les fichiers \textit{\textbf{tab1D.h}} et \textit{\textbf{tab1D.cpp}} dans le répertoire du projet.
	\item Exécutez le programme. 
	\item Expliquez le code fourni et sa structure. Que contiennent notamment les fichiers \textit{\textbf{.h}} et \textit{\textbf{.cpp}} ? 
\end{enumerate}

\centerline {\rule{.50\linewidth}{.25pt}} % Horizontal line
%-----------------------------------
\textbf{Exercice 2 / Maximum, somme et moyenne d'un tableau 1D}

Ajoutez les fonctions \textit{maxTabInt(...)}, \textit{sumTabInt(...)} et \textit{moyTabInt(...)} à la bibliothèque de fonctions sur les tableaux 1D, permettant respectivement de retourner le maximum, la somme et la moyenne d'un tableau d'entiers passé en argument.


\centerline {\rule{.50\linewidth}{.25pt}} % Horizontal line
%-----------------------------------
\textbf{Exercice 3 / Masques binaires}

Soit les variables \textbf{\lstinline|int a = 478;|} et \textbf{\lstinline|int k;|}.

\begin{enumerate}
	 \item Quel est le résultat des opérations \textbf{\lstinline|k = a << 3;|} et \textbf{\lstinline|k = a >> 2;|} ?
	 \item Quel est le résultat de l'opération \textbf{\lstinline|k = a & 0b11111111;|} ?
	 
\medskip
	 
Soit les variables \textbf{\lstinline|int k1 = (a >> 8) & 0b11111111;|} et \textbf{\lstinline|int k2 = (a) & 0b11111111;|}.

	\item Quel est le résultat de l'opération \textbf{\lstinline|k = (k1 << 8) + k2;|} ? Expliquez ces trois instructions.

\end{enumerate}


\centerline {\rule{.50\linewidth}{.25pt}} % Horizontal line
%-----------------------------------
\textbf{Exercice 4 / Chaine de caractères}

On se propose d'étudier le code \textit{\textbf{exo2\_4.cpp}} fourni.

\begin{enumerate}
	\item Testez ce code et expliquez comment est gérée une chaine de caractères en C/C++.
	\item Commentez les différents affichages de la variable \textbf{\lstinline|chaine[2]|}.
	\item Qu'arrive-t-il à la variable \textbf{\lstinline|chaine[3]|} stockée dans la variable \textbf{\lstinline|c|}.
	\item A quoi sert l'instruction \textbf{\lstinline|sprintf(...)|} ? Combien d'octets utilise la variable \textbf{\lstinline|chaine2|} ?
		
\end{enumerate}

\centerline {\rule{.50\linewidth}{.25pt}} % Horizontal line
%-----------------------------------
\textbf{Exercice 5 / Transmission d'informations}

On souhaite transmettre 5 données entières qui sont chacune comprise entre 0 et 179. 

\begin{enumerate}
	\item Proposez une solution pour les transmettre de manière asynchrone octet par octet de manière fiable, en utilisant le moins d'octets possible.
	\item Cette solution est-elle toujours la moins gourmande en octets dans le cas de 5 nombres entiers compris entre 0 et 3599 ?
\end{enumerate}

\centerline {\rule{.50\linewidth}{.25pt}} % Horizontal line
%-----------------------------------
\textbf{Exercice B1 / Echange de variables}

Écrire une fonction qui prend deux entiers A et B en paramètre, qui échange le contenu des variables A et B. 


\centerline {\rule{.50\linewidth}{.25pt}} % Horizontal line
%-----------------------------------
\textbf{Exercice B2 / Remplissage d'un tableau 1D}

Écrire un programme qui demande à l'utilisateur de saisir 10 entiers stockés dans un tableau. Le programme doit ensuite afficher l'indice du plus grand élément.


\centerline {\rule{.70\linewidth}{.25pt}} % Horizontal line
%----------------------------------------------------------------------------------------

\end{document} 