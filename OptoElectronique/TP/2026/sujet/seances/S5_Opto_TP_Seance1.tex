\newpage
%\pagestyle{empty}

\begin{minipage}[c]{.25\linewidth}
	\includegraphics[width=4cm]{images/Logo-LEnsE.png}
\end{minipage} \hfill
\begin{minipage}[c]{.4\linewidth}

\begin{center}
\vspace{0.3cm}
{\Large \textsc{Opto-Électronique}}

\medskip

5N-027-SCI \qquad \textbf{\Large TP Séance 1}

\end{center}
\end{minipage}\hfill

%%%%%%%%%%%%%%%%%%%%%%%%%%%%%%%%%%
\section{Objectif de la séance}

Lors de cette première séance, vous allez :

\begin{itemize}
	\item étudier un premier \textbf{montage de photodétection} et déterminer certaines de ses caractéristiques
	\item \textbf{caractériser statiquement le capteur} de ce montage, la photodiode
	\item \textbf{mettre en \oe{}uvre un montage d'amplification} basé sur un amplificateur linéaire intégré (ALI) et le \textbf{caractériser fréquentiellement} et \textbf{temporellement}
\end{itemize} 

\medskip

Vous allez également vous \textbf{familiariser avec l'utilisation des appareils de mesure} mis à
votre disposition au cours des séances de Travaux Pratiques d'Opto-Électronique et \textbf{mettre en \oe{}uvre des protocoles expérimentaux standards} en photonique.

%%%%%%%%%%%%%%%%%%%%%%%%%%%%%%%%%%
\section{Etude d'un montage simple de photodétection - \textit{Durée : 90 min}}

On s'intéresse au circuit de la figure~\ref{fig:schem_photo} avec $E = 5 \, V$ (tension continue), une photodiode de type \textbf{\texttt{SFH206K}} (documentation fournie dans le document annexe) et $R_{PHD} = 100 \, k\Omega$:

\begin{figure}[h!]
    \centering
\begin{circuitikz}
	\draw (0,0) to[battery2, invert] (0,3) -- (2,3);
	% fleche
	\draw (-0.5,0.3) edge[->] (-0.5,2.7);
	\node (Ein) at (-1,1.5){$E$};
	
	\draw (5,3) to[empty photodiode] (2,3);
	\draw (2,3) to[short, i = $I_{photo}$, current arrow scale=8] ++(0.5,0);

	\draw (5,3) to[R=$R_{PHD}$, *-*] (5,0) -- (0,0);
	\draw (5,0) to[short, -o] ++(1.5,0);
	\draw (5,3) to[short, -o] ++(1.5,0);
	\draw (5,0) node[ground](GND){};
	% fleche
	\draw (6.5,0.3) edge[->, green!40!black] (6.5,2.7); 
	\node[text=green!40!black] (US) at (7.1,1.5){$V_S$};

	\node (PhiE) at (3,2){$\Phi_e$};	
	
\end{circuitikz}
    \caption{Schéma du circuit de photodétection simple}
    \label{fig:schem_photo}
\end{figure}

On souhaite pouvoir remplir le tableau suivant (à reproduire dans votre cahier de laboratoire) :

\medskip

\begin{center}
\begin{tabular}{|c|c|c|c|c|}
  \hline
  Grandeur & Unité & Flux ambiant & Obscurité & Lampe de bureau \\
  \hline
  Intensité lumineuse $\Phi_e$ &  &  &  &  \\
  \hline
  Tension $V_S$ &  &  &  &  \\
  \hline
  Courant $I_{photo}$ &  &  &  &  \\
  \hline
\end{tabular}
\end{center}

\Quest Proposer des protocoles (avec les schémas associés incluant les appareils de mesure) pour remplir ce tableau.

\Quest Quel est le lien entre la tension $V_S$ et le flux lumineux mesuré $\Phi_e$ ?

\medskip

\Manip Câbler le montage ci-dessus, en incluant les instruments de mesure adéquats et relever les informations nécessaires pour le tableau précédent.

\Quest Chercher dans la documentation technique de la photodiode la valeur de la \textbf{sensibilité spectrale} et comparer le courant mesuré au courant théorique. Que pouvez-vous conclure ?

\Manip Visualiser la tension aux bornes de $R_{PHD}$ à l'aide d'un oscilloscope et comparer le signal observé au signal attendu. 


\clearpage
%%%%%%%%%%%%%%%%%%%%%%%%%%%%%%%%%%
%%%%%%%%%%%%%%%%%%%%%%%%%%%%%%%%%%
%%%%%%%%%%%%%%%%%%%%%%%%%%%%%%%%%%
\section{Etude statique d'une photodiode - \textit{Durée : 60 min}}

Le système de photodétection précédent inclus un \textbf{capteur particulier}, une photodiode, que nous allons à présent chercher à \textbf{caractériser statiquement}, c'est à dire de \textbf{tracer expérimentalement la loi mathématique} qui lie le courant traversant le dipôle et la différence de potentiel à ses bornes.

Nous utiliserons ici une méthode de tracer automatisée de la caractéristique statique.


%%%%%%%%%%%%%%%%%%%%%%%%%%%%%%%%%%
\subsection{Ressources}

\begin{itemize}
	\item Fiche : Diode / LED / Photodiode
	\item Fiche : Photodétection
	\item Protocole : Caractéristique statique d'un dipôle / Caractéristique Automatisée
\end{itemize}


\subsection{Photodiode \texttt{SFH206K}}

On utilisera une photodiode de type \textbf{\texttt{SFH206K}} (une partie de la documentation est fournie en annexe).

\Quest Rechercher et relever dans la documentation technique du constructeur de la photodiode \texttt{SFH206K} les valeurs intéressantes pour la mise en \oe{}uvre pratique (électrique et optique) d'un tel composant.


\subsection{Choix des appareils et des composants}

Dans le schéma proposé dans la rubrique \textbf{Caractéristique Automatisée} du tutoriel \textit{Caractéristique statique d'un dipôle}, une résistance $R_P$ est proposée comme protection en courant et une résistance $R_I$ comme élément de mesure du courant.

On choisira $R_P = 270\operatorname{\Omega}$ et $R_I = 10\operatorname{\Omega}$. \textit{Le calcul de $R_P$ sera étudié lors de la séance suivante.}

\Manip Relever la caractéristique $i=f(u)$ de cette photodiode, lorsqu'elle est plongée dans l'obscurité, pour des tensions $u$ positives ET négatives.

\Manip Relever la caractéristique $i=f(u)$ de cette photodiode, lorsqu'elle est soumise à un flux lumineux constant, pour des tensions $u$ positives ET négatives. Quelle précaution faut-il prendre lors de cette mesure ?

\Manip Déterminer les zones d'utilisation possible de ce composant. Quel modèle peut-on alors proposer pour ce capteur ? 

%%%%%%%%%%%%%%%%%%%%%%%%%%%%%%%%%%
\subsection{Validation des résultats}

\Real Préparer une présentation (3-4 min maximum) rassemblant les \textbf{schémas de mesure}, les \textbf{protocoles utilisés}, les \textbf{calculs réalisés}, les \textbf{résultats pertinents} obtenus ainsi qu'une \textbf{brève analyse} de ces derniers.

\medskip

\noindent \rule{\linewidth}{1pt}

\textbf{\Large Faire valider l'ensemble par un$\cdot$e encadrant$\cdot$e}

\medskip

Cette présentation devra clairement faire ressortir des preuves en lien avec :

\begin{itemize}
	\item les compétences \fbox{C3}, \fbox{C4} et \fbox{C5}
	\item les AAV du \fbox{bloc 1}
\end{itemize}



\clearpage
%%%%%%%%%%%%%%%%%%%%%%%%%%%%%%%%%%
%%%%%%%%%%%%%%%%%%%%%%%%%%%%%%%%%%
%%%%%%%%%%%%%%%%%%%%%%%%%%%%%%%%%%
\section{Etude fréquentielle d'un montage amplificateur - \textit{Durée : 120 min}}

On se propose d'étudier le circuit \textbf{amplificateur inverseur} dont le schéma est donné dans la figure~\ref{fig:schem_ampli} :

\begin{figure}[h!]
    \centering
\begin{circuitikz} 
	\node [op amp, fill=blue!10!white](A1) at (0,0){\texttt{AOP1}};
	\draw (A1.-) to[short] ++(-.5,0) coordinate(A) to[short] ++(0,1.5) coordinate(B) to[R=$R_2$] (B -| A1.out) to[short, -*] (A1.out);
	\draw (A1.-) to[short,-*] ++(-.5,0) coordinate(AA) to[R=$R_1$] ++(-2.5,0) coordinate(BB) to[short,-o] ++(-.5,0) coordinate(CC);
	\draw (A1.+) to[short] ++(0,-0.5) node[ground]{};
	\draw (A1.out) to[short,-o] ++(1,0) coordinate(D);
	\draw (-4.6,-1) edge[->,color={green!40!black}] (-4.6,0.3);
	\node[text={green!40!black}] (Ve) at (-5.1,-0.35){$V_e$}; 
	\draw (-4.6,-1.3)  to[open,-o] ++(0,0) node[ground](GND){};
	\draw (2.2,-1) edge[->, color={red}] (2.2,-0.3);
	\node[text={red}] (Vs) at (1.7,-0.6){$V_s$}; 
	\draw (2.2,-1.3)  to[open,-o] ++(0,0) node[ground](GND){};
	
\end{circuitikz}
    \caption{Schéma d'un circuit amplificateur inverseur}
    \label{fig:schem_ampli}
\end{figure}

Ce montage utilise un \textbf{amplificateur linéaire intégré (ALI)}. Pour ce TP, on choisira un ALI de type \textbf{\texttt{TL081}}.

\medskip

On peut montrer que la fonction de transfert théorique d'un tel montage vaut :

\[
\boxed{\frac{V_s}{V_e} = - \frac{R_2}{R_1}}
\]

\medskip

On souhaite \textbf{vérifier que cette loi reste valable} quelque soit la fréquence et l'amplitude du signal d'entrée.

\medskip

Pour cela, on souhaite pouvoir remplir le tableau suivant (à reproduire dans votre cahier de laboratoire) :

\medskip

\begin{center}
\begin{tabular}{|c|c|c|c|c|c|c|}
  \hline
  Gain (dB) & [A]mplification & $R_1$ & $R_2$ & [B]ande [P]assante & Produit [A].[BP] & $\Delta{}T_{95\%}$\\
  \hline
  $12$ &  &  &  & & & \\
  \hline
  $32$ &  &  &  & & & \\
  \hline
\end{tabular}
\end{center}

où l'amplification correspond à $\frac{V_s}{V_e}$, $R_1$ et $R_2$ les valeurs des résistances choisies, la bande-passante à $-3\operatorname{dB}$ et $\Delta{}T_{95\%}$ est le temps de réponse à 95\%.



\subsection{Ressources}

\begin{itemize}
	\item Fiche : Amplificateur Linéaire Intégré
	\item Fiche : Amplificateur Linéaire Intégré / Modèle
	\item Fiche : Régime Harmonique
	\item Fiche : Filtrage / Analyse Harmonique / Ordre 1
	\item Protocole : Réponse en fréquence d'un système linéaire / Procédure "classique"
	\item Protocole : Réponse indicielle d'un système linéaire
\end{itemize}


\clearpage
\subsection{Alimentation symétrique}
Certains composants, notamment les amplificateurs linéaires intégrés, sont capables de traiter des différences de potentiel positives et négatives. Pour cela, il est nécessaire de les alimenter de manière symétrique, c'est-à-dire, avec deux sources de tension fournissant des tensions opposées, souvent notées \texttt{+VCC}, pour l'alimentation positive, et \texttt{-VCC}, pour l'alimentation négative. 

\Quest A partir de la documentation technique, noter le câblage du composant \textbf{\texttt{TL081}} et les tensions d'alimentation maximales. 

\Manip Réaliser une \textbf{alimentation symétrique +10V / -10V} à partir des alimentations stabilisées disponibles et mettre en place un système de contrôle de ces tensions.

\Quest Quelles précautions faudra-t-il prendre sur la tension d'entrée de ce montage ?

%%%%%%%%%%%%%%%%%%%%%%%%%%%%%%%%%%
\subsection{Réponse en fréquence}

\Quest Quelles valeurs de résistances choisir pour obtenir un gain de $12~\operatorname{dB}$ ? \textit{La somme des résistances doit être comprise entre $10\operatorname{k\Omega}$ et $50\operatorname{k\Omega}$.}

\Manip Réaliser le montage précédent et alimenter le avec l'alimentation symétrique réalisée.

\Manip Tracer le \textbf{diagramme de Bode en gain} de ce système pour des fréquences allant de $100~\operatorname{Hz}$ à $1~\operatorname{MHz}$, à l'aide de mesure réalisée à l'oscilloscope. 

\Quest Quelle était la réponse en fréquence attendue théoriquement ? 

\Manip Mesurer la bande-passante à $-3\operatorname{dB}$ de ce montage.

%%%%%%%%%%%%%%%%%%%%%%%%%%%%%%%%%%
\noindent \rule{\linewidth}{1pt}

\Manip Modifier la résistance $R_1$ pour obtenir un gain de $32~\operatorname{dB}$.

\Manip Tracer le \textbf{diagramme de Bode en gain} de ce système pour des fréquences allant de $100~\operatorname{Hz}$ à $1~\operatorname{MHz}$, à l'aide de mesure réalisée à l'oscilloscope. 

\Manip Mesurer la bande-passante à $-3\operatorname{dB}$ de ce nouveau montage.

\Quest Préciser alors le modèle à utiliser pour ce montage. 


%%%%%%%%%%%%%%%%%%%%%%%%%%%%%%%%%%
\subsection{Réponse indicielle}

\Manip Pour les deux montages précédents, visualiser la réponse indicielle.

\Manip Mesurer le temps de réponse à 95\%.

\Quest Quel est le lien entre ce temps et la bande-passante mesurée dans la partie précédente ?


%%%%%%%%%%%%%%%%%%%%%%%%%%%%%%%%%%
%%%%%%%%%%%%%%%%%%%%%%%%%%%%%%%%%%
\subsection{Cahier de laboratoire / Check-List}

\begin{itemize}[label=$\square$]
	\item tableau comparatif rempli
	\item protocoles utilisés (avec schéma de câblage, paramètres des instruments de mesure et étapes expérimentales)
	\item diagrammes de Bode légendés
	\item captures d'écran d'oscilloscope des réponses indicielles
	\item analyse des résultats
	\item modélisation du système
\end{itemize}




%%%%%%%%%%%%%%%%%%%%%%%%%%%%%%%%%%%%%%%%%%%%%%%%%%%%%%%%%%%%%%%%%%%%%%%%%%%%%%%%%%%%%