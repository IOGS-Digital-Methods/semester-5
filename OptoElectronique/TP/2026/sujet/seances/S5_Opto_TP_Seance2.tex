\newpage
%\pagestyle{empty}

\begin{minipage}[c]{.25\linewidth}
	\includegraphics[width=4cm]{images/LEnsE_IOGS.jpg}
\end{minipage} \hfill
\begin{minipage}[c]{.4\linewidth}

\begin{center}
\vspace{0.3cm}
{\Large \textsc{Opto-Électronique}}

\medskip

5N-027-SCI \qquad \textbf{\Large TP Séance 2}

\end{center}
\end{minipage}\hfill

%%%%%%%%%%%%%%%%%%%%%%%%%%%%%%%%%%
\section{Objectif de la séance}

Lors de cette seconde séance, vous allez :

\begin{itemize}
	\item mettre en place un \textbf{banc de caractérisation} d'un montage de photodétection
	\begin{itemize}
		\item \textbf{caractériser statiquement une LED}	
		\item réaliser un \textbf{émetteur de flux lumineux} basé sur une source à LED
	\end{itemize} 
	
	\item mettre en \oe{}uvre un \textbf{montage simple de photodétection} et le \textbf{caractériser fréquentiellement} et \textbf{temporellement}
\end{itemize} 


%%%%%%%%%%%%%%%%%%%%%%%%%%%%%%%%%%
\subsection{Premier modèle du montage de photodétection}

Les premiers résultats obtenus lors de la précédente séance autour du montage simple de photodétection ont permis de conclure que ce type de montage permettait d'\textbf{obtenir une tension proportionnelle au flux lumineux}.

\medskip

On peut montrer que la fonction de transfert théorique d'un tel montage vaut :

\[
\boxed{V_s = R_{PHD} \cdot I_{photo} = R_{PHD} \cdot k \cdot \Phi_{e}}
\]

où $k$ est la sensibilité de la photodiode et $\Phi_e$ le flux lumineux à mesurer.

%%%%%%%%%%%%%%%%%%%%%%%%%%%%%%%%%%
\subsection{Validité du modèle}

On souhaite \textbf{vérifier que cette loi reste valable} quelque soit la fréquence et l'amplitude du flux lumineux d'entrée.

\medskip

Pour cela, on souhaite pouvoir remplir le tableau suivant (à reproduire dans votre cahier de laboratoire) et \textbf{mesurer la réponse en fréquence de ce montage} :

\medskip

\begin{center}
\begin{tabular}{|c|c|c|c|c|c|}
  \hline
  $R_{PHD}$ & $|V_s|_{MAX}$ & [B]ande [P]assante & [BP]/$R_{PHD}$ \\
  \hline
  $10\operatorname{k\Omega}$ & & & \\
  \hline
  $100\operatorname{k\Omega}$ & & & \\
  \hline
  $1\operatorname{M\Omega}$ & & & \\
  \hline
\end{tabular}
\end{center}

où $|V_s|_{MAX}$ est l'amplitude maximale du signal de sortie, [BP] est la bande-passante à $-3\operatorname{dB}$ du système et [BP]/$R_{PHD}$ le rapport de la bande-passante sur la valeur de la résistance $R_{PHD}$.

\bigskip

%%%%%%%%%%%%%%%%%%%%%%%%%%%%%%%%%%
\subsection{Nécessité d'une source lumineuse paramètrable}

Afin de pouvoir étudier le montage de photodétection pour différentes fréquences, il est \textbf{indispensable} d'avoir à disposition \textbf{une source lumineuse dont la fréquence du flux lumineux est modifiable}.


\clearpage
%%%%%%%%%%%%%%%%%%%%%%%%%%%%%%%%%%
%%%%%%%%%%%%%%%%%%%%%%%%%%%%%%%%%%
%%%%%%%%%%%%%%%%%%%%%%%%%%%%%%%%%%
\section{Etude statique d'une LED - \textit{Durée : 60 min}}

Nous allons chercher à \textbf{caractériser statiquement} une source lumineuse de type \textbf{LED}, c'est à dire \textbf{tracer expérimentalement la loi mathématique} qui lie le courant traversant le dipôle et la différence de potentiel à ses bornes.


%%%%%%%%%%%%%%%%%%%%%%%%%%%%%%%%%%
\subsection{Ressources}

\begin{itemize}
	\item Fiche : Diode / LED / Photodiode
	\item Protocole : Caractéristique statique d'un dipôle / Caractéristique Manuelle
\end{itemize}

%%%%%%%%%%%%%%%%%%%%%%%%%%%%%%%%%%
\subsection{LED Rouge}

On utilisera une LED Rouge de type \textbf{\texttt{Kingbright L-1503ID}} (une partie de la documentation est fournie en annexe).

\Quest Rechercher et relever, dans la documentation technique du constructeur de la LED Rouge, les valeurs intéressantes pour la mise en \oe{}uvre pratique (électrique et optique) d'un tel composant.


\subsection{Choix des appareils et des composants}

Dans le schéma proposé dans la rubrique \textbf{Caractéristique manuelle} du tutoriel \textit{Caractéristique statique d'un dipôle}, une résistance $R_P$ est proposée comme protection en courant.

\Quest Comment choisir cette résistance et comment régler les différents appareils de mesure?

\Manip Relever la caractéristique $i=f(u)$ de cette LED pour des tensions $u$ positives ET négatives. 

\Manip Déterminer les zones d'utilisation possible de ce composant. 

\medskip

%%%%%%%%%%%%%%%%%%%%%%%%%%%%%%%%%%
\subsection{Cahier de laboratoire / Check-List}

\begin{itemize}[label=$\square$]
	\item protocoles utilisés (avec schéma de câblage, paramètres des instruments de mesure et étapes expérimentales)
	\item courbe de la caractéristique statique
	\item analyse de la courbe
\end{itemize}

%%%%%%%%%%%%%%%%%%%%%%%%%%%%%%%%%%
%%%%%%%%%%%%%%%%%%%%%%%%%%%%%%%%%%
\clearpage
\section{Réalisation d'un émetteur lumineux - \textit{Durée : 60 min}}

Afin de pouvoir caractériser en fréquence le montage de photodétection, nous allons avoir besoin d'un émetteur lumineux dont il est possible de \textbf{contrôler la fréquence du flux lumineux émis} (voir circuit sur la figure~\ref{fig:schem_emit_trans}).

\begin{figure}[h!]
    \centering
\begin{circuitikz}
	\draw (-2,0) to[short, o-*] (-1, 0) -- (0, 0) to[R=$R_{LED}$] (0, 2);
	\draw (-2, 4.5) to[short, o-] (0, 4.5) to[led, l_=$LED$] (0, 2);
	\draw (-1,0) node[ground](GND){};
	\draw (-2,0.3) edge[->, red!40!black] (-2,4.2);
	\node[text=red!40!black] (Vin) at (-2.5,2.25){$V_e$};
	
	\draw[->, thick, blue!40!black, decorate, decoration={snake, amplitude=1mm, segment length=4mm}] (0.8,3.25) -- (2.2,3.25);
	\node[text=blue!40!black] (Phi) at (1.5,4){$\Phi_e$};

	\draw (7,0) to[battery2, invert] (7,4.5) -- (3,4.5);
	% fleche
	\draw (7.5,0.3) edge[->, thick] (7.5,4.2);
	\node (Ein) at (8,2.25){$E$};
	
	\draw (3,2) to[empty photodiode, -] (3,4.5);
	\draw (3,2.8) to[short, i = $I_{photo}$, current arrow scale=8] ++(0,-0.5);

	\draw (3,2) to[R=$R_{PHD}$, *-] (3,0) to[short, -*] (4,0) -- (7,0);
	\draw (3,2) to[short, -o] ++(1.5,0);
	\draw (4,0) node[ground](GND){};
	% fleche
	\draw (4.5,0.3) edge[->, thick, green!40!black] (4.5,1.7); 
	\node[text=green!40!black] (US) at (5,1){$V_S$};
\end{circuitikz}
    \caption{Schéma du circuit émetteur (à gauche) et du circuit de photodétection simple (à droite)}
    \label{fig:schem_emit_trans}
\end{figure}

où $\Phi_e$ est le flux lumineux résultant de l'émetteur.

\medskip

On souhaite un \textbf{flux lumineux sinusoïdal} dont il est possible de contrôler la fréquence de modulation. \textit{Attention, on ne parle pas ici de la variation de la longueur d'onde de la source lumineuse mais bien de la modulation du flux lumineux en amplitude.}

\Quest Quel signal doit-on appliquer sur $V_e$ ? Quels sont les paramètres à donner à ce signal (amplitude, valeur moyenne...) pour obtenir un flux lumineux sinusoïdal ?

\Quest A quoi correspond la résistance $R_{LED}$ ? Comment vérifier que le flux est sinusoïdal (sans utiliser le montage de photodétection proposé puisqu'on ne connaît pas sa réponse en fréquence) ?

\Manip Réaliser le système d'émission et vérifier son bon fonctionnement.


%%%%%%%%%%%%%%%%%%%%%%%%%%%%%%%%%%
\subsection{Cahier de laboratoire / Check-List}

\begin{itemize}[label=$\square$]
	\item paramètres utilisés pour le générateur $V_e$
	\item protocole de vérification de la forme du flux lumineux
	\item validation du fonctionnement (capture d'écran d'oscilloscope)
\end{itemize}


%%%%%%%%%%%%%%%%%%%%%%%%%%%%%%%%%%
\subsection{Montage de photodétection - Rappel}

Dans le circuit précédent, la partie de droite correspond au montage de photodétection que nous allons par la suite chercher à caractériser.

\Quest A quoi sert la tension $E$ ? De quelle nature doit-elle être ?

\Quest Quelle fonction de transfert cherche-t-on à caractériser sur ce montage de photodétection ? 

\Quest A-t-on accès directement à la valeur du flux lumineux $\Phi_e$ ? De quoi dépend le flux lumineux reçu par la photodiode (en lien avec celui émis par la LED) ? Quelles seront alors les précautions à prendre lors des prochaines mesures ?



\clearpage
%%%%%%%%%%%%%%%%%%%%%%%%%%%%%%%%%%
%%%%%%%%%%%%%%%%%%%%%%%%%%%%%%%%%%
%%%%%%%%%%%%%%%%%%%%%%%%%%%%%%%%%%
\section{Réponse en fréquence du montage - \textit{Durée : 120 min}}

On souhaite à présent \textbf{vérifier la validité du modèle proposé initialement} pour le montage simple de photodétection pour des flux lumineux modulés à différentes fréquences.

Pour cela, on placera le système émetteur devant le montage de photodétection à caractériser (voir schéma de la section précédente).

\subsection{Ressources}

\begin{itemize}
	\item Fiche : Photodétection
	\item Fiche : Filtrage / Analyse Harmonique / Ordre 1
	\item Protocole : Réponse en fréquence d'un système linéaire / Procédure "classique"
	\item Protocole : Mesure de bande-passante
	\item Protocole : Réponse indicielle d'un système linéaire
\end{itemize}



%%%%%%%%%%%%%%%%%%%%%%%%%%%%%%%%%%
\subsection{Réponse en fréquence}

\Manip Placer le montage émetteur face au montage de photodétection (la LED en face de la partie sensible de la photodiode). Utiliser une résistance $R_{PHD} = 100\operatorname{k\Omega}$.

\Manip Tracer le \textbf{diagramme de Bode en gain} de ce système pour des fréquences allant de $100~\operatorname{Hz}$ à $1~\operatorname{MHz}$, à l'aide de mesure réalisée à l'oscilloscope.

\Quest Quelle était la réponse en fréquence attendue théoriquement ? 

\Manip Mesurer la bande-passante à $-3\operatorname{dB}$ de ce montage.

\medskip

\Manip Refaire le tracé du diagramme de Bode et la mesure de la bande-passante pour des résistances $R_{PHD} = 10\operatorname{k\Omega}$ et $R_{PHD} = 1\operatorname{M\Omega}$

\Quest Conclure sur le modèle à utiliser pour ce montage. 



%%%%%%%%%%%%%%%%%%%%%%%%%%%%%%%%%%
\subsection{Validation des résultats}

\Real Préparer une présentation (3-4 min maximum) rassemblant les \textbf{schémas de mesure}, les \textbf{protocoles utilisés}, les \textbf{calculs réalisés}, les \textbf{résultats pertinents} obtenus ainsi qu'une \textbf{brève analyse} de ces derniers.

\medskip

\noindent \rule{\linewidth}{1pt}

\textbf{\Large Faire valider l'ensemble par un$\cdot$e encadrant$\cdot$e}

\medskip

Cette présentation devra clairement faire ressortir des preuves en lien avec :

\begin{itemize}
	\item les compétences \fbox{C3}, \fbox{C4} et \fbox{C5}
	\item les AAV du \fbox{bloc 2}
\end{itemize}



%%%%%%%%%%%%%%%%%%%%%%%%%%%%%%%%%%%%%%%%%%%%%%%%%%%%%%%%%%%%%%%%%%%%%%%%%%%%%%%%%%%%%