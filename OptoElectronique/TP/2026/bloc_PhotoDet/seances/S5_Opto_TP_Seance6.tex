\newpage
%\pagestyle{empty}

\begin{minipage}[c]{.25\linewidth}
	\includegraphics[width=4cm]{images/LEnsE_IOGS.jpg}
\end{minipage} \hfill
\begin{minipage}[c]{.4\linewidth}

\begin{center}
\vspace{0.3cm}
{\Large \textsc{Opto-Électronique}}

\medskip

5N-027-SCI \qquad \textbf{\Large TP Séance 6}

\end{center}
\end{minipage}\hfill

\bigskip

%%%%%%%%%%%%%%%%%%%%%%%%%%%%%%%%%%
\noindent \rule{\linewidth}{1pt}

Lors de cette dernière séance, vous aurez le choix parmi trois propositions.

Ce choix est laissé au libre arbitre des binômes en fonction de leurs besoins de consolider les connaissances expérimentales et théoriques sur la photodétection, de découvrir une application de la photodétection ou de mettre en oeuvre un système de régulation numérique.

%%%%%%%%%%%%%%%%%%%%%%%%%%%%%%%%%%
\section{Sujet au choix}

Au cours de cette dernière séance, vous aurez à choisir parmi l'une des trois possibilités suivantes :

\begin{itemize}[label=$\square$]
	\item \textbf{consolider vos acquis} sur les travaux pratiques précédents autour de la \textbf{photodétection}
	\item mettre en place une \textbf{transmission de données par la lumière} basée sur de la modulation d'amplitude (idéalement à 2 binômes)
	\begin{itemize}[label=$\blacktriangleright$]
		\item Analyse spectrale, mise en \oe{}uvre d'un émetteur et d'un photodétecteur
	\end{itemize}	
	
	\item mettre en place un système de \textbf{régulation de température} incluant des aspects de traitement numérique de l'information
	\begin{itemize}[label=$\blacktriangleright$]
		\item Etude d'un système numérique, mise en \oe{}uvre d'un capteur de température, étude de la commande d'un moteur à courant continu
	\end{itemize}
	
\end{itemize}

\medskip

Les deux dernières propositions se basent partiellement sur des maquettes (en nombre limité) à étudier puis à compléter pour obtenir l'application souhaitée.

\bigskip

\textit{Les sujets \textbf{Transmission par la lumière} et \textbf{Régulation de température} sont disponibles sur le site du LEnsE.}




%%%%%%%%%%%%%%%%%%%%%%%%%%%%%%%%%%%%%%%%%%%%%%%%%%%%%%%%%%%%%%%%%%%%%%%%%%%%%%%%%%%%%