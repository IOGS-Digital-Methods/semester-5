\newpage
\pagestyle{empty}

\begin{minipage}[c]{.25\linewidth}
	\includegraphics[width=4cm]{images/Logo-LEnsE.png}
\end{minipage} \hfill
\begin{minipage}[c]{.4\linewidth}

\begin{center}
\vspace{0.3cm}
{\Large \textsc{Opto-Electronique}}

\medskip

5N-027-SCI \qquad \textbf{\Large TP Bloc 3}

\end{center}
\end{minipage}\hfill

\vspace{0.5cm}

\noindent \rule{\linewidth}{1pt}

{\noindent\Large \textbf{Mission 3.1} / Réaliser un circuit d'émission à LED} 

\vspace{-0.5cm}

\begin{center}
\noindent \rule{\linewidth}{1pt}

Durée conseillée : 60 min / Séance 3 ou 4

\vspace{-0.2cm}
\noindent \rule{\linewidth}{1pt}
\end{center}

%%%%%%%%%%%%%%%%%%%%%%%%%%%%%%%%%%
\section{Objectif de la mission}
\label{mission31}

On se propose de \textbf{réaliser un circuit d'émission} d'un flux lumineux pour \textbf{transmettre un signal sinusoïdal} et de valider son fonctionnement.

%%%%%%%%%%%%%%%%%%%%%%%%%%%%%%%%%%
%%%%%%%%%%%%%%%%%%%%%%%%%%%%%%%%%%
%%%%%%%%%%%%%%%%%%%%%%%%%%%%%%%%%%
\section{Ressources}

Vous pouvez utiliser la fiche résumée suivante : 

\begin{itemize}
	\item \hyperref[fiche:Led]{Fiche : Diode / LED / Photodiode}
\end{itemize}


\subsection{LED Rouge}

On utilisera une LED rouge de type \textbf{Kingbrigth L-1503ID} ($V_f = 2\operatorname{V}$, $I_{fmax} = 25~mA$).

\subsection{Mise en oeuvre}

\Quest Proposer un montage permettant de faire fonctionner la LED dans sa zone d'émission. Comment choisir la résistance de protection de la LED ? 

\Quest Quelle forme de tension appliquée ? Comment régler les différents appareils de mesure pour éviter de dégrader la LED et garantir un flux lumineux sinusoïdal ?

\Manip Réaliser le montage.

\Quest Proposer une méthode pour valider le bon fonctionnement de votre montage (sans utiliser de système basé sur la mesure du flux lumineux...).

\Manip Valider le fonctionnement du montage.

%%%%%%%%%%%%%%%%%%%%%%%%%%%%%%%%%%
%%%%%%%%%%%%%%%%%%%%%%%%%%%%%%%%%%
\section{Livrables}


Une \textbf{fiche de manipulation} en ligne (partagée dans le cahier de laboratoire) rappelant :

\begin{itemize}
	\item les choix technologiques réalisés (schémas de câblage et valeur des composants)
	\item le choix des réglages des instruments (schémas de mesure)
	\item les courbes obtenues
\end{itemize}

Une \textbf{analyse} du résultat obtenu.


%%%%%%%%%%%%%%%%%%%%%%%%%%%%%%%%%%
%%%%%%%%%%%%%%%%%%%%%%%%%%%%%%%%%%