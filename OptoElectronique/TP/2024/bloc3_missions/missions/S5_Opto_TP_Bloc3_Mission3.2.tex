\newpage
\pagestyle{empty}

\begin{minipage}[c]{.25\linewidth}
	\includegraphics[width=4cm]{images/Logo-LEnsE.png}
\end{minipage} \hfill
\begin{minipage}[c]{.4\linewidth}

\begin{center}
\vspace{0.3cm}
{\Large \textsc{Opto-Electronique}}

\medskip

5N-027-SCI \qquad \textbf{\Large TP Bloc 3}

\end{center}
\end{minipage}\hfill

\vspace{0.5cm}

\noindent \rule{\linewidth}{1pt}

{\noindent\Large \textbf{Mission 3.2} / Mettre en \oe{}uvre et caractériser un circuit de photodétection simple} 

\vspace{-0.5cm}

\begin{center}
\noindent \rule{\linewidth}{1pt}

Durée conseillée : 90 min / Séance 3 ou 4

\vspace{-0.2cm}
\noindent \rule{\linewidth}{1pt}
\end{center}

%%%%%%%%%%%%%%%%%%%%%%%%%%%%%%%%%%
\section{Objectif de la mission}
\label{mission32}

On se propose de \textbf{réaliser un circuit de photodétection "simple"} et de \textbf{caractériser ses performances dynamiques} (réponse en fréquence notamment).

%%%%%%%%%%%%%%%%%%%%%%%%%%%%%%%%%%
%%%%%%%%%%%%%%%%%%%%%%%%%%%%%%%%%%
%%%%%%%%%%%%%%%%%%%%%%%%%%%%%%%%%%
\section{Ressources}

\begin{itemize}
	\item \hyperref[fiche:Photodetect]{Fiche : Photodétection}
\end{itemize}


\subsection{Circuit à étudier}

\begin{multicols}{2}
On se propose d'analyser le circuit ci-contre, avec $R_{PHD} = 100\operatorname{k\Omega}$.


\Quest Quel est le lien entre $V_S$ et $I_{photo}$ ? Puis entre $I_{photo}$ et le flux lumineux capté par la photodiode $\Phi_e$ ?

\Quest Si le flux lumineux reçu est sinusoïdal, quelle sera la forme de la tension de sortie $V_S$ ? Quelles sont les limites en amplitude de ce montage ?

\Manip Réaliser le montage. Placer-le devant le montage d'émission réalisé dans la mission 3.1.

\medskip

\columnbreak

\begin{center}
\begin{circuitikz}
	\draw (0,0) to[battery2, invert] (0,4.5) -- (2,4.5);
	% fleche
	\draw (-0.5,0.3) edge[->] (-0.5,4.2);
	\node (Ein) at (-1,2.25){$E$};
	
	\draw (2,2) to[empty photodiode, *-*] (2,4.5) to[short, -o] ++(1.5,0);
	\draw (2,2.8) to[short, i = $I_{photo}$, current arrow scale=8] ++(0,-0.5);

	\draw (2,2) to[R=$R_{PHD}$, *-*] (2,0) -- (0,0);
	\draw (2,2) to[short, -o] ++(1.5,0);
	\draw (1,0) node[ground](GND){};
	\draw (2,0) to[short, -o] ++(1.5,0);
	% fleche
	\draw (3.5,2.3) edge[->] (3.5,4.2); \node (Ud) at (4,3.25){$u_d$};
	% fleche
	\draw (3.5,0.3) edge[->, green!40!black] (3.5,1.7); \node[text=green!40!black] (US) at (4,1){$V_S$};
	\draw (3.5,0) to[short, -o] ++(0,0);
\end{circuitikz}
\end{center}

\end{multicols}

\Manip Relever les formes du courant $I_f$ dans la LED et de la tension $V_S$ pour diverses valeurs de fréquence du signal d'entrée.

\Quest Quel est l'impact de $E$ sur la tension de sortie $V_S$ ?

\Quest Quelle est la forme théorique de la réponse en fréquence de ce montage ?




\subsection{Réponse en fréquence}

\Manip Tracer l'allure rapide de la réponse en fréquence en gain de ce système pour des fréquences allant de $100~\operatorname{Hz}$ à $1~\operatorname{MHz}$.

\Manip Faire une mesure de la bande-passante du montage.

\Manip Reproduire les deux précédentes étapes pour des valeurs $R_{PHD} = 10\operatorname{k\Omega}$ et $R_{PHD} = 1\operatorname{M\Omega}$. 

\Quest Quel est alors l'impact de $R_{PHD}$ sur les performances du montage ?

\subsection{Réponse indicielle}

\Manip Tracer la réponse indicielle de ce système pour les 3 valeurs de résistance proposées précédemment : $R_{PHD} = 10\operatorname{k\Omega}$, $R_{PHD} = 100\operatorname{k\Omega}$ et $R_{PHD} = 1\operatorname{M\Omega}$.

\Manip Mesurer le temps de réponse à 95\% dans chacun des cas.

\Quest Comparer les résultats obtenus avec les mesures de bande-passante.

%%%%%%%%%%%%%%%%%%%%%%%%%%%%%%%%%%
%%%%%%%%%%%%%%%%%%%%%%%%%%%%%%%%%%
\section{Modélisation}

Afin d'expliquer le phénomène observé précédemment, il est possible d'affiner le modèle utilisé pour l'étude du montage précédent en prenant en compte les éléments "perturbateurs".

Voici le modèle plus complet du montage étudié précédemment.

\begin{center}
\begin{circuitikz}
	% blocs
	\fill[green,fill opacity=.1] (-0.4,-0.2) rectangle (0.5,5.2);
	\fill[blue,fill opacity=.1] (4.8,2.2) rectangle (10.6,-0.2);	
	\fill[orange,fill opacity=.1] (1.5,5.2) rectangle (5.1,2.3);
	% legende blocs
	%\fill[green,fill opacity=.1] (9,5.5) rectangle (11,4.7);
	%\node (pol) at (10,5.1){Polarisation};
	%\fill[orange,fill opacity=.1] (9,4.5) rectangle (11,3.7);
	%\node (pol) at (10,4.1){Photodiode};	
	%\fill[blue,fill opacity=.1] (9,3.5) rectangle (11,2.7);
	%\node (pol) at (10,3.1){Affichage};	
	% circuit
	\draw (-0.5,0.3) edge[->] (-0.5,4.7); \node (Ein) at (-1,2.5){$E$};
	\draw (0,0) to[battery2, invert] (0,5) -- (2,5) to[I, *-*] (2,2.5) -- (2,2) to[R=$R_{PHD}$, *-*] (2,0) -- (0,0);
	\draw (2,3.3) to[short, i = $ I_{photo}$, current arrow scale=10] ++(0,-0.5);
	\draw (2,5) -- (3.5,5) to[C=$C_{PHD}$] (3.5,2.5) -- (2,2.5);
	\draw (1.3,2.8) edge[->] (1.3,4.7); 	\node (Ud) at (0.7,3.5){$u_d$};
		
	\draw (3.5,0.3) edge[->, green!40!black] (3.5,1.7); \node[text=green!40!black] (US) at (4,1){$V_S$};
	
	\draw (2,2) -- (5.5,2) to[C=$C_{cab}$, *-*] (5.5,0) -- (2,0);
	\draw (5.5,2) -- (7.5,2) to[C=$C_{osc}$, *-*] (7.5,0) -- (5.5,0);
	\draw (7.5,2) -- (9.5,2) to[R=$R_{osc}$] (9.5,0) -- (7.5,0);

\end{circuitikz}
\end{center}


\Quest A quoi correspondent les différents éléments présents ?

\Quest A partir des mesures réalisées précédemment, comment remonter aux valeurs du modèle précédent ? Donner les valeurs des différents éléments qu'il est possible de calculer.

%%%%%%%%%%%%%%%%%%%%%%%%%%%%%%%%%%
%%%%%%%%%%%%%%%%%%%%%%%%%%%%%%%%%%
\section{Livrables}


Une \textbf{fiche de manipulation} en ligne (partagée dans le cahier de laboratoire) rappelant :

\begin{itemize}
	\item les protocoles de mesure et de réglage (schémas de mesure, de câblage)
	\item les tableaux de mesures et les courbes obtenues
\end{itemize}

Une \textbf{analyse} du résultat obtenu, en particulier le lien entre le gain, la bande-passante du montage, le temps de réponse et la valeur de $R_{PHD}$.

Un \textbf{tableau comparatif} de ces différents éléments pour les 3 valeurs de $R_{PHD}$.

%%%%%%%%%%%%%%%%%%%%%%%%%%%%%%%%%%
%%%%%%%%%%%%%%%%%%%%%%%%%%%%%%%%%%