\newpage
%\pagestyle{empty}

\begin{minipage}[c]{.25\linewidth}
	\includegraphics[width=4cm]{images/Logo-LEnsE.png}
\end{minipage} \hfill
\begin{minipage}[c]{.4\linewidth}

\begin{center}
\vspace{0.3cm}
{\Large \textsc{Opto-Électronique}}

\medskip

5N-027-SCI \qquad \textbf{\Large TP Bloc 1}

\end{center}
\end{minipage}\hfill

\vspace{0.5cm}

\noindent \rule{\linewidth}{1pt}

{\noindent\Large \rule[-7pt]{0pt}{30pt} \textbf{Mission 1.1} / Tracer la caractéristique $i = f(u)$ d'une photodiode} 

\noindent \rule{\linewidth}{1pt}

\vspace{-0.5cm}

\begin{center}

Durée conseillée : 60 min / Séance 1

\end{center}

%%%%%%%%%%%%%%%%%%%%%%%%%%%%%%%%%%
\section{Objectif de la mission}
\label{mission11}

On se propose de \textbf{caractériser une photodiode}, c'est à dire de \textbf{tracer expérimentalement la loi mathématique} qui lie le courant traversant le dipôle et la différence de potentiel à ses bornes.


%%%%%%%%%%%%%%%%%%%%%%%%%%%%%%%%%%
%%%%%%%%%%%%%%%%%%%%%%%%%%%%%%%%%%
%%%%%%%%%%%%%%%%%%%%%%%%%%%%%%%%%%
\section{Ressources}

Vous pouvez utiliser les fiches résumées suivantes : 

\begin{itemize}
	\item \hyperref[fiche:Led]{Fiche : Diode / LED / Photodiode}
	\item \hyperref[fiche:Photodetect]{Fiche : Photodétection}
\end{itemize}


\subsection{Photodiode \texttt{SFH206K}}

On utilisera une photodiode de type \textbf{\texttt{SFH206K}} (une partie de la \hyperref[doc:phdSFH206K]{documentation} est fournie en annexe).

\Quest Rechercher et relever dans la documentation technique du constructeur de la photodiode \texttt{SFH206K} les valeurs intéressantes pour la mise en \oe{}uvre pratique (électrique et optique) d'un tel composant.


\subsection{Méthode conventionnelle}

Vous utiliserez une méthode classique de l'instrumentation pour relever les points de la courbe $i = f(u)$. Vous pourrez vous inspirer de la partie \textbf{Caractéristique manuelle} du tutoriel \hyperref[ressource:CaracStat]{Caractériser un dipôle}.

\subsection{Choix des appareils et des composants}

Dans le schéma proposé dans le tutoriel \textbf{Caractéristique manuelle} du tutoriel \hyperref[ressource:CaracStat]{Caractériser un dipôle}, une résistance $R_P$ est proposée comme protection en courant.

\Quest Comment choisir cette résistance et comment régler les différents appareils de mesure?

\Manip Relever la caractéristique $i=f(u)$ de cette photodiode, lorsqu'elle est plongée dans l'obscurité, pour des tensions $u$ positives ET négatives.


%%%%%%%%%%%%%%%%%%%%%%%%%%%%%%%%%%
\section{Livrables}


Une \textbf{fiche de manipulation} en ligne (partagée dans le cahier de laboratoire) rappelant :

\begin{itemize}
	\item les protocoles de mesure et de réglage (schémas de mesure, de câblage)
	\item les tableaux de mesures et les courbes obtenues
\end{itemize}

Une \textbf{analyse} du résultat obtenu, en particulier sur les valeurs de courant obtenues pour une tension négative en présence ou non de photons.

%%%%%%%%%%%%%%%%%%%%%%%%%%%%%%%%%%%%%%%%%%%%%%%%%%%%%%%%%%%%%%%%%%%%%%%%%%%%%%%%%%%%%