\newpage
\pagestyle{empty}

\begin{minipage}[c]{.25\linewidth}
	\includegraphics[width=4cm]{images/Logo-LEnsE.png}
\end{minipage} \hfill
\begin{minipage}[c]{.4\linewidth}

\begin{center}
\vspace{0.3cm}
{\Large \textsc{Opto-Électronique}}

\medskip

5N-027-SCI \qquad \textbf{\Large TP Bloc 1}

\end{center}
\end{minipage}\hfill

\vspace{0.5cm}

\noindent \rule{\linewidth}{1pt}

{\noindent\Large \rule[-7pt]{0pt}{30pt} \textbf{Mission 1.3} / Tracer la caractéristique $i = f(u)$ d'un dipôle de manière automatisée} 

\noindent \rule{\linewidth}{1pt}

\vspace{-0.5cm}

\begin{center}

Durée conseillée : 60 min / Séance 2

\end{center}

%%%%%%%%%%%%%%%%%%%%%%%%%%%%%%%%%%
\section{Objectif de la mission}
\label{mission13}

On souhaite \textbf{caractériser une photodiode} (dans le domaine du visible), c'est à dire \textbf{tracer expérimentalement la loi mathématique} qui lie le courant traversant le dipôle et la différence de potentiel à ses bornes de manière plus automatisée que lors de la mission 1.1.

On souhaite voir également l'évolution de cette caractéristique en fonction de l'éclairement de la photodiode.


%%%%%%%%%%%%%%%%%%%%%%%%%%%%%%%%%%
%%%%%%%%%%%%%%%%%%%%%%%%%%%%%%%%%%
%%%%%%%%%%%%%%%%%%%%%%%%%%%%%%%%%%
\section{Ressources}

Vous pouvez utiliser les fiches résumées suivantes : 

\begin{itemize}
	\item \hyperref[fiche:Led]{Fiche : Diode / LED / Photodiode}
	\item \hyperref[fiche:Photodetect]{Fiche : Photodétection}
\end{itemize}


\subsection{Méthode automatisée}

Vous utiliserez cette fois-ci une méthode plus rapide pour relever une allure de la courbe $i = f(u)$. Vous pourrez vous inspirer de la partie \textbf{Caractéristique automatique} du tutoriel \hyperref[ressource:CaracStat]{Caractériser un dipôle}.

\Manip Relever la caractéristique statique de la photodiode dans les deux cas suivants:

\begin{itemize}
	\item dans l'obscurité
	\item en présence d'un flux lumineux (fourni par une source externe)
\end{itemize}

\Manip Mesurer, à l'aide d'un luxmètre, le flux lumineux émis par la source externe utilisée.



%%%%%%%%%%%%%%%%%%%%%%%%%%%%%%%%%%
\section{Livrables}


Une \textbf{fiche de manipulation} en ligne (partagée dans le cahier de laboratoire) rappelant :

\begin{itemize}
	\item les protocoles de mesure et de réglage (schémas de mesure, de câblage)
	\item les tableaux de mesures et les courbes obtenues, %
	copies d'écran d'oscilloscope en particulier %[FaB]
\end{itemize}

Une \textbf{analyse} du résultat obtenu et des zones d'utilisation possible d'une photodiode. On s'intéressera en particulier à la zone qui permet de mesurer un flux lumineux.

%%%%%%%%%%%%%%%%%%%%%%%%%%%%%%%%%%%%%%%%%%%%%%%%%%%%%%%%%%%%%%%%%%%%%%%%%%%%%%%%%%%%%
