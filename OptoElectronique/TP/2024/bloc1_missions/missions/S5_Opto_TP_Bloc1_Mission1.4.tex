
\newpage
\pagestyle{empty}

\begin{minipage}[c]{.25\linewidth}
	\includegraphics[width=4cm]{images/Logo-LEnsE.png}
\end{minipage} \hfill
\begin{minipage}[c]{.4\linewidth}

\begin{center}
\vspace{0.3cm}
{\Large \textsc{Opto-Électronique}}

\medskip

5N-027-SCI \qquad \textbf{\Large TP Bloc 1}

\end{center}
\end{minipage}\hfill

\vspace{0.5cm}

\noindent \rule{\linewidth}{1pt}

{\noindent\Large \rule[-7pt]{0pt}{30pt}  \textbf{Mission 1.4} / Tracer la caractéristique $i = f(u)$ d'une LED} 

\noindent \rule{\linewidth}{1pt}

\vspace{-0.5cm}

\begin{center}

Durée conseillée : 30 min / Séance 3 ou 4

\end{center}

%%%%%%%%%%%%%%%%%%%%%%%%%%%%%%%%%%
\section{Objectif de la mission}
\label{mission14}

On souhaite \textbf{caractériser une LED rouge}, c'est à dire \textbf{tracer expérimentalement la loi mathématique} qui lie le courant traversant le dipôle et la différence de potentiel à ses bornes, afin de déterminer un point de fonctionnement idéal pour transmettre un signal sinusoïdal.


%%%%%%%%%%%%%%%%%%%%%%%%%%%%%%%%%%
%%%%%%%%%%%%%%%%%%%%%%%%%%%%%%%%%%
%%%%%%%%%%%%%%%%%%%%%%%%%%%%%%%%%%
\section{Ressources}

Vous pouvez utiliser la fiche résumée suivante : 

\begin{itemize}
	\item \hyperref[fiche:Led]{Fiche : Diode / LED / Photodiode}
\end{itemize}

\subsection{LED Rouge}

On utilisera une LED rouge de type \texttt{Kingbrigth L-1503ID} (une partie de la \hyperref[doc:ledRouge]{documentation} est fournie en annexe).

\Quest Rechercher et relever dans la documentation technique du constructeur de la LED les valeurs intéressantes pour la mise en oeuvre pratique (électrique et optique) d'un tel composant.

\subsection{Méthode automatisée}

Vous utiliserez cette fois-ci une méthode plus rapide pour relever une allure de la courbe $i = f(u)$. Vous pourrez vous inspirer de la partie \textbf{Caractéristique automatique} du tutoriel \hyperref[ressource:CaracStat]{Caractériser un dipôle}.

\Quest Comment choisir la résistance de protection de la LED ? Comment régler les différents appareils de mesure pour éviter de dégrader la LED ?

\Manip Relever la caractéristique statique de la LED.


%%%%%%%%%%%%%%%%%%%%%%%%%%%%%%%%%%
\section{Livrables}


Une \textbf{fiche de manipulation} en ligne (partagée dans le cahier de laboratoire) rappelant :

\begin{itemize}
	\item les protocoles de mesure et de réglage (schémas de mesure, de câblage)
	\item les tableaux de mesures et les courbes obtenues
\end{itemize}

Une \textbf{analyse} du résultat obtenu.

Quelques lignes expliquant :

\begin{itemize}
	\item dans quelle zone la LED peut-être utilisée pour moduler la lumière émise,
	\item les précautions à prendre pour obtenir une modulation sinusoïdale du flux lumineux.
\end{itemize}

%%%%%%%%%%%%%%%%%%%%%%%%%%%%%%%%%%%%%%%%%%%%%%%%%%%%%%%%%%%%%%%%%%%%%%%%%%%%%%%%%%%%%
