\newpage
\pagestyle{empty}

\begin{minipage}[c]{.25\linewidth}
	\includegraphics[width=4cm]{images/Logo-LEnsE.png}
\end{minipage} \hfill
\begin{minipage}[c]{.4\linewidth}

\begin{center}
\vspace{0.3cm}
{\Large \textsc{Opto-Électronique}}

\medskip

5N-027-SCI \qquad \textbf{\Large TP Bloc 1}

\end{center}
\end{minipage}\hfill

\vspace{0.5cm}

\noindent \rule{\linewidth}{1pt}

{\noindent\Large  \rule[-7pt]{0pt}{30pt}  \textbf{Mission 1.2} / Mesurer le courant généré par une photodiode en mode capteur} 

\noindent \rule{\linewidth}{1pt}

\vspace{-0.5cm}

\begin{center}

Durée conseillée : 30 min / Séance 1

\end{center}

%%%%%%%%%%%%%%%%%%%%%%%%%%%%%%%%%%
\section{Objectif de la mission}
\label{mission12}

On souhaite \textbf{mesurer le courant généré par une photodiode} (dans le domaine du visible) à l'aide d'un ampèremètre pour différentes valeurs d'éclairement.


%%%%%%%%%%%%%%%%%%%%%%%%%%%%%%%%%%
%%%%%%%%%%%%%%%%%%%%%%%%%%%%%%%%%%
%%%%%%%%%%%%%%%%%%%%%%%%%%%%%%%%%%
\section{Ressources}

Vous pouvez utiliser les fiches résumées suivantes : 

\begin{itemize}
	\item \hyperref[fiche:Led]{Fiche : Diode / LED / Photodiode}
	\item \hyperref[fiche:Photodetect]{Fiche : Photodétection}
\end{itemize}

\subsection{Utilisation d'un luxmètre}

Afin d'avoir une donnée de comparaison d'éclairement ambiant de la salle de travaux pratiques, un luxmètre est mis à votre disposition.

Vous pourrez ainsi comparer certaines données du constructeur avec vos résultats...


\Manip A l'aide d'un luxmètre, relever des valeurs d'éclairement :
\begin{itemize}
	\item dans l'obscurité,
	\item de la paillasse éclairée avec l'éclairage ambiant,
	\item de la paillasse éclairée avec une source supplémentaire (lampe de bureau ou lampe du téléphone portable).
\end{itemize}

\Manip Relever les valeurs du courant $i$ obtenu en sortie de la photodiode dans les trois conditions d'éclairage précédentes, pour une tension directe de l'ordre de $1\operatorname{V}$ et pour une tension inverse de $5\operatorname{V}$ (tension mesurée dans l'obscurité).

\Quest Dans quel cas est-il plus intéressant d'utiliser cette photodiode comme capteur de photons ?


%%%%%%%%%%%%%%%%%%%%%%%%%%%%%%%%%%
\section{Livrables}


Une \textbf{fiche de manipulation} en ligne (partagée dans le cahier de laboratoire) rappelant :

\begin{itemize}
	\item les protocoles de mesure et de réglage (schémas de mesure, de câblage)
	\item les tableaux de mesures obtenues
\end{itemize}

Une \textbf{analyse} du résultat obtenu et une comparaison entre les valeurs expérimentale et théorique obtenues pour le courant en présence du flux lumineux ambiant.

%%%%%%%%%%%%%%%%%%%%%%%%%%%%%%%%%%%%%%%%%%%%%%%%%%%%%%%%%%%%%%%%%%%%%%%%%%%%%%%%%%%%%
