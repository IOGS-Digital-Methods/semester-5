\newpage
\pagestyle{empty}

\begin{minipage}[c]{.25\linewidth}
	\includegraphics[width=4cm]{images/Logo-LEnsE.png}
\end{minipage} \hfill
\begin{minipage}[c]{.4\linewidth}

\begin{center}
\vspace{0.3cm}
{\Large \textsc{Opto-Electronique}}

\medskip

\textbf{\Large Ressources}

\end{center}
\end{minipage}\hfill

\vspace{0.5cm}

\noindent \rule{\linewidth}{1pt}
\section{Tracer la réponse en fréquence d'un système linéaire}
\label{ressource:RepFreq}


%%%%%%%%%%%%%%%%%%%%%%%%%%%%%%%%%%%%%%%%%%%%%%%%%%%%%%%%%%%%%%%%%%%%%%%%%%%%%%%%
%%%%%

La réponse en fréquence décrit la manière dont \textbf{un système réagit à différentes fréquences d'entrée}. Elle indique comment l'amplitude et la phase d'un signal sont modifiées lorsqu'il passe à travers un système ou un circuit, en fonction de la fréquence du signal.

Cette étude traduit le \textbf{comportement harmonique} d'un circuit, c'est à dire sa réponse à une excitation (en tension) sinusoïdale.

Cette fonction n'est définie que dans le cas de circuits linéaires. La tension de sortie est dans ce cas sinusoïdale et de même fréquence que le signal d'entrée.

\subsection*{Objectif}

Obtenir l'allure de la \textbf{courbe du gain} et éventuellement de celle \textbf{du déphasage} apportés par le circuit en fonction de la fréquence du signal sinusoïdal placé en entrée.

Le \textbf{diagramme de Bode} est une manière spécifique de représenter la réponse en fréquence. Il s'agit d'une représentation graphique qui se compose de deux parties distinctes :

\begin{itemize}
	\item diagramme de Bode en gain (gain en $\operatorname{dB}$),
	\item diagramme de Bode en phase.
\end{itemize}

La courbe est souvent tracée avec une échelle logarithmique en fréquence. 

\subsection*{Protocole d'étude}

\begin{itemize}
	\item Régler un \textbf{générateur de fonction} (ou GBF) pour avoir un \textbf{signal sinusoïdal} à sa sortie, avec une amplitude compatible avec les limites du circuit à tester.
	\item Relier le générateur de fonction à la fois à l'entrée du circuit et à une des entrées de l'oscilloscope.
	\item Relier la tension de sortie à une deuxième voie de l'oscilloscope.
	\item S'assurer que le signal de sortie est bien sinusoïdal avant d'aller plus loin. Dans le cas contraire, le système ne fonctionne pas de manière linéaire (amplitude trop élevée en entrée par exemple qui entraine une saturation en sortie...)
\end{itemize}


\section{Procédure classique}

Un premier balayage rapide en fréquence permet de \textbf{repérer les gammes de fréquences d'intérêt}.

Une analyse du \textbf{comportement du circuit pour les valeurs extrêmes de fréquences} (sur la phase et l'amplitude) apporte les informations sur le comportement asymptotique de la réponse en fréquence. Ce sont les \textbf{2 premiers points de mesure}.

\textbf{3 à 5 mesures supplémentaires} sont ensuite suffisantes :
\begin{itemize}
	\item l'une à la fréquence caractéristique du circuit, qui peut être : 
	\begin{itemize}
		\item la fréquence centrale d'un circuit passe-bande, 
		\item la bande passante à $-3\operatorname{dB}$ pour un circuit passe-bas ou passe-haut,
		\item la fréquence d'un déphasage particulier (en général la fréquence pour laquelle le déphasage apporté est égal à la moitié du déphasage maximal que le circuit peut apporter )
	\end{itemize}
	\item les autres de part et d'autres de cette fréquence caractéristique, à une octave au dessous (à la fréquence moitié) et une octave au dessus ( à la fréquence double)
\end{itemize}


\subsection*{Mesure du gain du circuit}

Il existe \textbf{deux solutions pour déterminer la valeur du gain} :

\begin{itemize}
	\item Utiliser les mesures automatiques de l'oscilloscope, pour relever l'amplitude du signal d'entrée et du signal de sortie, et utiliser le logiciel pour convertir le gain en $\operatorname{dB}$
	\item Utiliser le multimètre en dB-mètre (voir documentation annexe des instruments de mesure).
\end{itemize}

\subsection*{Mesure du déphasage}

Certains oscilloscopes proposent des mesures automatiques du déphasage. En leur absence, une mesure aux curseurs du décalage temporel $\Delta{}T$ entre les deux tensions (entrée et sortie) permet de remonter au déphasage $\Delta{}\phi$ par la formule :

$$\Delta{}\phi = +/- \frac{\Delta{}T}{T} \cdot 2\pi$$

où T est la période du signal.

Il est important de déterminer lequel des deux signaux est en avance sur l'autre, afin de donner un signe au déphasage apporté par le circuit. Si la tension de sortie est en retard sur la tension d'entrée, le déphasage est négatif.


\section{Allure rapide}

Il existe également une \textbf{méthode automatique} pour obtenir l'allure de la réponse en fréquence du système, selon le modèle de GBF que vous possédez.

En effet, certains d'entre eux sont capables de réaliser automatiquement un balayage en fréquence.

Pour les GBF \textbf{\texttt{Agilent}} (des salles de TP d'électronique, par exemple), il faut utiliser au préalable sélectionner un \textbf{signal sinusoïdal}, de n'importe quelle fréquence mais d'amplitude et de valeur moyenne (offset) compatible avec le système à étudier (si ALI/AOP, vérifiez que le signal de sortie ne sature pas, par exemple).

Puis sélectionner ensuite le menu \textbf{Sweep} du GBF. Se référer ensuite à la documentation du GBF fournie sur vos paillasses pour les réglages.


Il est ensuite possible de synchroniser l'oscilloscope avec le GBF en utilisant la sortie \textbf{Sync}, qui fournit un signal rectangulaire de même période que le balayage, connectée à l'une des entrées de l'oscilloscope (EXT si on ne souhaite que synchroniser) et en réglant les paramètres du déclenchement de l'oscilloscope.