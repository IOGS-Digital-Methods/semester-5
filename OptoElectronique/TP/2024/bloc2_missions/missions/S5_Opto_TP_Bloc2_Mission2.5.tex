\newpage
\pagestyle{empty}

\begin{minipage}[c]{.25\linewidth}
	\includegraphics[width=4cm]{images/Logo-LEnsE.png}
\end{minipage} \hfill
\begin{minipage}[c]{.4\linewidth}

\begin{center}
\vspace{0.3cm}
{\Large \textsc{Opto-Electronique}}

\medskip

5N-027-SCI \qquad \textbf{\Large TP Bloc 2}

\end{center}
\end{minipage}\hfill

\vspace{0.5cm}

\noindent \rule{\linewidth}{1pt}

{\noindent\Large \textbf{Mission 2.5} / Tracer la réponse indicielle d'un système} 

\noindent \rule{\linewidth}{1pt}

\vspace{-0.5cm}

\begin{center}

Durée conseillée : 60 min / Séance 2

\end{center}

%%%%%%%%%%%%%%%%%%%%%%%%%%%%%%%%%%
\section{Objectif de la mission}
\label{mission25}

On se propose de \textbf{tracer la réponse indicielle} d'un montage et de montrer l'impact du gain du montage sur le temps de réponse du montage.

\textit{On reprendra le montage de la Mission 2.4.}


%%%%%%%%%%%%%%%%%%%%%%%%%%%%%%%%%%
%%%%%%%%%%%%%%%%%%%%%%%%%%%%%%%%%%
%%%%%%%%%%%%%%%%%%%%%%%%%%%%%%%%%%
\section{Ressources}

Vous pouvez utiliser les fiches résumées suivantes : 

\begin{itemize}
	\item \hyperref[fiche:ALIModele]{Fiche : Amplificateur Linéaire Intégré / Modèle}
	\item \hyperref[fiche:AnHaOrdre1]{Fiche : Analyse Harmonique / Ordre 1}
\end{itemize}


%%%%%%%%%%%%%%%%%%%%%%%%%%%%%%%%%%
\subsection{Réponse indicielle}

La réponse indicielle fournit des informations essentielles sur les \textbf{caractéristiques dynamiques} d'un système.

Elle est aussi appelée \textbf{réponse à un échelon} unitaire. Elle correspond à la sortie d'un système lorsqu'un signal de type échelon est appliqué en entrée.

Un échelon unitaire est une fonction qui passe de 0 à 1 à un instant donné, généralement considéré à $t = 0$. 

Elle permet d'analyser le comportement transitoire et le comportement à l'état stable d'un système. Par exemple, on peut observer le temps de montée, le temps de stabilisation, le dépassement (ordre supérieur à 2), et l'erreur en régime permanent.

%%%%%%%%%%%%%%%%%%%%%%%%%%%%%%%%%%
\subsection{Lien avec la réponse impulsionnelle et la réponse en fréquence}

La réponse indicielle est liée à la réponse impulsionnelle $h(t)$ d'un système par la relation suivante :

$$u_S(t) = \int_{0}^t h(x)dx$$

Cela signifie que la réponse indicielle est l'intégrale de la réponse impulsionnelle.

On rappelle également que la réponse en fréquence, que l'on peut modéliser par la fonction de transfert d'un circuit en fonction de la fréquence $H(j\omega)$ est la transformée de Fourier de la réponse impulsionnelle $h(t)$.


%%%%%%%%%%%%%%%%%%%%%%%%%%%%%%%%%%
\subsection{Travail à réaliser}

\Manip Tracer la réponse indicielle du système et relever le temps de réponse à $95\%$ pour la valeur de rapport $\frac{R_2}{R_1}$ permettant d'obtenir un gain dans la bande-passante de $25\operatorname{dB}$.

\Manip Tracer la réponse indicielle de ce système pour 2 autres valeurs du rapport $\frac{R_2}{R_1}$. Calculer également la nouvelle valeur de la fréquence caractéristique du montage ($f_c = R_2 \cdot $). Relever le temps de réponse à $95\%$ pour ces deux valeurs de rapport $\frac{R_2}{R_1}$.

\medskip

\textit{La somme des résistances $R_1 + R_2$ doit être comprise entre $10\operatorname{k\Omega}$ et $50\operatorname{k\Omega}$}.

%%%%%%%%%%%%%%%%%%%%%%%%%%%%%%%%%%
\section{Livrables}


Une \textbf{fiche de manipulation} en ligne (partagée dans le cahier de laboratoire) rappelant :

\begin{itemize}
	\item le protocole de mesure et de réglage (schémas de mesure, de câblage)
	\item les mesures du temps de réponse à $95\%$
	\item les captures d'écran de l'oscilloscope pour les différentes réponses indicielles
\end{itemize}

Une \textbf{analyse} du résultat obtenu et une étude du lien entre la bande-passante et le temps de réponse à 95\%.

%%%%%%%%%%%%%%%%%%%%%%%%%%%%%%%%%%%%%%%%%%%%%%%%%%%%%%%%%%%%%%%%%%%%%%%%%%%%%%%%%%%%%