\newpage
\pagestyle{empty}

\begin{minipage}[c]{.25\linewidth}
	\includegraphics[width=4cm]{images/Logo-LEnsE.png}
\end{minipage} \hfill
\begin{minipage}[c]{.4\linewidth}

\begin{center}
\vspace{0.3cm}
{\Large \textsc{Opto-Electronique}}

\medskip

5N-027-SCI \qquad \textbf{\Large TP Bloc 2}

\end{center}
\end{minipage}\hfill

\vspace{0.5cm}

\noindent \rule{\linewidth}{1pt}

{\noindent\Large \textbf{Mission 2.4} / Mesurer l'écart de phase entre deux signaux} 

\noindent \rule{\linewidth}{1pt}

\vspace{-0.5cm}

\begin{center}

Durée conseillée : 30 min / Séance 2

\end{center}

%%%%%%%%%%%%%%%%%%%%%%%%%%%%%%%%%%
\section{Objectif de la mission}
\label{mission24}

On se propose de \textbf{mesurer la phase} séparant deux signaux électriques à l'aide de l'oscilloscope.

\textit{On ajoutera un condensateur de $100\operatorname{nF}$ en parallèle de la résistance $R_2$ du montage de la Mission 2.1.}


%%%%%%%%%%%%%%%%%%%%%%%%%%%%%%%%%%
%%%%%%%%%%%%%%%%%%%%%%%%%%%%%%%%%%
%%%%%%%%%%%%%%%%%%%%%%%%%%%%%%%%%%
\section{Ressources}

Vous pouvez utiliser les fiches résumées suivantes : 

\begin{itemize}
	\item \hyperref[fiche:ALIModele]{Fiche : Amplificateur Linéaire Intégré / Modèle}
	\item \hyperref[fiche:AnHaOrdre1]{Fiche : Analyse Harmonique / Ordre 1}
\end{itemize}

%%%%%%%%%%%%%%%%%%%%%%%%%%%%%%%%%%
\subsection{Montage filtre actif passe-bas}

\begin{multicols}{2}

On se propose d'étudier le montage suivant. \textit{Tous les potentiels sont référencés par rapport à la masse.}

La fonction de transfert de ce montage, en supposant l'ALI idéal, est la suivante : 

$$\frac{V_s}{V_e} = -\frac{R_2}{R_1} \cdot \frac{1}{1 + j R_2 C \omega}$$

\medskip

\textit{On conservera les valeurs de $R_1$ et de $R_2$ du montage de la Mission 2.1, permettant d'obtenir un gain de $25\operatorname{dB}$.}

\columnbreak

\begin{center}
\begin{circuitikz} 
	\node [op amp, fill=blue!10!white](A1) at (0,0){\texttt{ALI1}};
	\draw (A1.-) to[short] ++(-.5,0) coordinate(A) to[short] ++(0,1.5) coordinate(B) to[R, l^=$R_2$] (B -| A1.out) coordinate(RR) to[short, -*] (A1.out);
	\draw (A1.-) to[short,-*] ++(-.5,0) coordinate(AA) to[R, l_=$R_1$] ++(-2.5,0) coordinate(BB) to [short,l_=${V_e}$, -o] ++(-1,0) coordinate(CC);

	\draw (B) to[short,*-] ++ (0, 1.5) coordinate(RC) to[C, l=$C$] (RC -| A1.out) to[short,-*] (RR);
	\draw (A1.+) to[short] ++(-.5,0) coordinate(AA2) node[cground]{};

	\draw (A1.out) to [short,l=${V_S}$, -o] ++(1,0) coordinate(D);
	
\end{circuitikz}
\end{center}
\end{multicols}

Ce montage a une fréquence caractéristique $f_c = \frac{1}{2 \cdot \pi \cdot R_2 \cdot C}$.

\Manip Vérifier expérimentalement que la bande-passante à $-3\operatorname{dB}$ vaut approximativement $f_c$.


%%%%%%%%%%%%%%%%%%%%%%%%%%%%%%%%%%
\subsection{Déphasage}

En régime harmonique (à même fréquence), deux ondes sinusoïdales peuvent avoir des phases initiales différentes.

Soient $u_1(t) = A_1 \cdot sin(2\pi f t + \varphi_1)$ et $u_2(t) = A_2 \cdot sin(2\pi f t + \varphi_2)$, le déphasage de l'une par rapport à l'autre à l'instant $t$ vaut : $\Delta\varphi = (2\pi f t + \varphi_2) - (2\pi f t + \varphi_1) = \varphi_2 - \varphi_1$

\medskip

Si $\Delta\varphi$ est positif, l'onde 2 est en avance de phase par rapport à l'onde 1. Sinon, l'onde 2 est en retard de phase par rapport à l'onde 1.

%%%%%%%%%%%%%%%%%%%%%%%%%%%%%%%%%%
\subsection{Phase et ordre d'un filtre}

Lorsqu'on étudie des systèmes linéaires de type filtre, il est intéressant de relever le déphasage entre le signal de sortie et le signal d'entrée pour différents points remarquables :

\begin{itemize}
	\item à la fréquence caractéristique du système, le déphasage est égal à $k\cdot \pi/4$ où $k$ est un entier correspondant à l'ordre du filtre
	\item loin de cette fréquence caractéristique (au moins une décade avant et après), pour vérifier le caractère inverseur d'un système par exemple.
\end{itemize}


%%%%%%%%%%%%%%%%%%%%%%%%%%%%%%%%%%
\subsection{Travail à réaliser}

\Manip Mesurer l'écart de phase entre les signaux $V_S$ et $V_e$ pour des fréquences égales à $f_c / 10$, $f_c$ et $10 f_c$ (où $f_c$ est la fréquence caractéristique du système).

\Quest Déterminer l'ordre de ce filtre.

%%%%%%%%%%%%%%%%%%%%%%%%%%%%%%%%%%
\section{Livrables}


Une \textbf{fiche de manipulation} en ligne (partagée dans le cahier de laboratoire) rappelant :

\begin{itemize}
	\item le protocole de mesure et de réglage (schémas de mesure, de câblage)
	\item les mesures des phases
\end{itemize}

Une \textbf{analyse} du résultat obtenu.

%%%%%%%%%%%%%%%%%%%%%%%%%%%%%%%%%%%%%%%%%%%%%%%%%%%%%%%%%%%%%%%%%%%%%%%%%%%%%%%%%%%%%