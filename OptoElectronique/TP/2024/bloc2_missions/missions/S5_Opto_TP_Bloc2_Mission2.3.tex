\newpage
\pagestyle{empty}

\begin{minipage}[c]{.25\linewidth}
	\includegraphics[width=4cm]{images/Logo-LEnsE.png}
\end{minipage} \hfill
\begin{minipage}[c]{.4\linewidth}

\begin{center}
\vspace{0.3cm}
{\Large \textsc{Opto-Electronique}}

\medskip

5N-027-SCI \qquad \textbf{\Large TP Bloc 2}

\end{center}
\end{minipage}\hfill

\vspace{0.5cm}

\noindent \rule{\linewidth}{1pt}

{\noindent\Large \rule[-7pt]{0pt}{30pt}  \textbf{Mission 2.3} / Tracer l'allure rapide de la réponse en fréquence} 

\noindent \rule{\linewidth}{1pt}

\vspace{-0.5cm}

\begin{center}

Durée conseillée : 60 min / Séance 2

\end{center}

%%%%%%%%%%%%%%%%%%%%%%%%%%%%%%%%%%
\section{Objectif de la mission}
\label{mission23}

On se propose de \textbf{tracer rapidement l'allure} de la réponse en fréquence d'un montage à amplificateur linéaire intégré de type inverseur (voir montage en Mission 2.1) et de montrer l'impact du gain du montage sur la bande-passante.


%%%%%%%%%%%%%%%%%%%%%%%%%%%%%%%%%%
%%%%%%%%%%%%%%%%%%%%%%%%%%%%%%%%%%
%%%%%%%%%%%%%%%%%%%%%%%%%%%%%%%%%%
\section{Ressources}

Vous pouvez utiliser les fiches résumées suivantes : 

\begin{itemize}
	\item \hyperref[fiche:ALIModele]{Fiche : Amplificateur Linéaire Intégré / Modèle}
	\item \hyperref[fiche:AnHaOrdre1]{Fiche : Analyse Harmonique / Ordre 1}
\end{itemize}


%%%%%%%%%%%%%%%%%%%%%%%%%%%%%%%%%%
\subsection{Réponse en fréquence de ce montage}

\Manip Reprendre le montage de la mission 2.1.

\Manip Tracer l'allure rapide de la réponse en fréquence en gain de ce système pour des fréquences allant de $10\operatorname{Hz}$ à $1\operatorname{MHz}$ et pour différentes valeurs de rapport $\frac{R_2}{R_1}$ (\textit{au moins 3 valeurs différentes}).

\Manip Relever à l'aide du \textit{marqueur en fréquence} les valeurs approchées de la fréquence de coupure.

Vous pouvez vous aider des ressources proposées à la fin du document : \hyperref[ressource:RepFreq]{Tracer la réponse en fréquence d'un système linéaire} (Partie \textit{Allure "rapide"}).
\medskip

\textit{La somme des résistances $R_1 + R_2$ doit être comprise entre $10\operatorname{k\Omega}$ et $50\operatorname{k\Omega}$}.

%%%%%%%%%%%%%%%%%%%%%%%%%%%%%%%%%%
\section{Livrables}


Une \textbf{fiche de manipulation} en ligne (partagée dans le cahier de laboratoire) rappelant :

\begin{itemize}
	\item le protocole de mesure et de réglage (schémas de mesure, de câblage)
	\item des captures d'écran de l'oscilloscope pour divers rapport $\frac{R_2}{R_1}$ contenant une mesure approchée de la fréquence de coupure
\end{itemize}

Une \textbf{analyse} du résultat obtenu et un bilan sur l'impact du gain du montage sur la bande-passante.

%%%%%%%%%%%%%%%%%%%%%%%%%%%%%%%%%%%%%%%%%%%%%%%%%%%%%%%%%%%%%%%%%%%%%%%%%%%%%%%%%%%%%