\newpage
\pagestyle{empty}

\begin{minipage}[c]{.25\linewidth}
	\includegraphics[width=4cm]{images/Logo-LEnsE.png}
\end{minipage} \hfill
\begin{minipage}[c]{.4\linewidth}

\begin{center}
\vspace{0.3cm}
{\Large \textsc{Opto-Electronique}}

\medskip

5N-027-SCI \qquad \textbf{\Large TP Bloc 2}

\end{center}
\end{minipage}\hfill

\vspace{0.5cm}

\noindent \rule{\linewidth}{1pt}

{\noindent\Large \rule[-7pt]{0pt}{30pt}  \textbf{Mission 2.1} / Tracer la réponse en fréquence d'un montage amplificateur inverseur et déterminer ses limites} 

\noindent \rule{\linewidth}{1pt}

\vspace{-0.5cm}

\begin{center}

Durée conseillée : 90 min / Séance 1

\end{center}

%%%%%%%%%%%%%%%%%%%%%%%%%%%%%%%%%%
\section{Objectif de la mission}
\label{mission21}

On se propose de \textbf{caractériser en fréquence} un montage à amplificateur linéaire intégré de type inverseur, c'est à dire de \textbf{tracer expérimentalement la loi mathématique} qui lie le gain du montage à la fréquence du signal injecté.

%%%%%%%%%%%%%%%%%%%%%%%%%%%%%%%%%%
%%%%%%%%%%%%%%%%%%%%%%%%%%%%%%%%%%
%%%%%%%%%%%%%%%%%%%%%%%%%%%%%%%%%%
\section{Ressources}

Vous pouvez utiliser les fiches résumées suivantes : 

\begin{itemize}
	\item \hyperref[fiche:ALI]{Fiche : Amplificateur Linéaire Intégré}
	\item \hyperref[fiche:ALIModele]{Fiche : Amplificateur Linéaire Intégré / Modèle}
	\item \hyperref[fiche:RegimeHarmonique]{Fiche : Régime Harmonique}
	\item \hyperref[fiche:AnHaOrdre1]{Fiche : Analyse Harmonique / Ordre 1}
\end{itemize}



\subsection{Amplificateur Linéaire Intégré}

Les \textbf{amplificateurs linéaires intégrés} (ALI) ou \textbf{amplificateurs opérationnels} (AOP) sont des composants actifs très largement utilisés dans le domaine de l'électronique pour \textbf{amplifier et mettre en forme} les signaux provenant de capteurs. 

\medskip

Nous allons étudier ici les deux principales limitations d'un ALI (ou circuit à AOP) en régime linéaire :

\begin{itemize}
	\item[\scriptsize$\blacksquare$] la \textbf{bande passante} qui est la limitation en fréquence du circuit à ALI ;	
	\item[\scriptsize$\blacksquare$] le \textbf{\textit{slew-rate}} ou vitesse de balayage maximale (voir \hyperref[parslewrate]{paragraphe sur le slew-rate}).
\end{itemize}

\medskip


%%%%%%%%%%%%%%%%%%%%%%%%%%%%%%%%%%%%%%%%%%%%%%%%%%%%%%%%%%%%%%%%%%%%
 
\Quest Rechercher, dans la documentation technique du composant \texttt{TL081}, les valeurs du \textit{slew-rate} et de la bande passante pour un gain unitaire (notée aussi produit "gain $\times$ bande~passante", GBW).

%%%%%%%%%%%%%%%%%%%%%%%%%%%%%%%%%%
\subsection{Montage amplificateur inverseur}

\begin{multicols}{2}

On se propose d'étudier le montage suivant. \textit{Tous les potentiels sont référencés par rapport à la masse.}

La fonction de transfert de ce montage, en supposant l'ALI idéal, est la suivante : 

$$\frac{V_s}{V_e} = -\frac{R_2}{R_1}$$

\medskip

\Quest Quelles valeurs de résistances choisir pour obtenir un gain de $25~\operatorname{dB}$ ? \textit{La somme des résistances doit être comprise entre $10\operatorname{k\Omega}$ et $50\operatorname{k\Omega}$.}

\columnbreak

\begin{center}
\begin{circuitikz} 
	\node [op amp, fill=blue!10!white](A1) at (0,0){\texttt{ALI1}};
	\draw (A1.-) to[short] ++(-.5,0) coordinate(A) to[short] ++(0,1.5) coordinate(B) to[R, l^=$R_2$] (B -| A1.out) coordinate(RR) to[short, -*] (A1.out);
	\draw (A1.-) to[short,-*] ++(-.5,0) coordinate(AA) to[R, l_=$R_1$] ++(-2.5,0) coordinate(BB) to [short,l_=${V_e}$, -o] ++(-1,0) coordinate(CC);
	\draw (A1.+) to[short] ++(-.5,0) coordinate(AA2) node[cground]{};
	\draw (A1.out) to [short,l=${V_S}$, -o] ++(1,0) coordinate(D);
	
\end{circuitikz}
\end{center}
\end{multicols}

%%%%%%%%%%%%%%%%%%%%%%%%%%%%%%%%%%
\subsection{Alimentation symétrique}
Certains composants, notamment les amplificateurs linéaires intégrés, sont capables de traiter des différences de potentiel positives et négatives. Pour cela, il est nécessaire de les alimenter de manière symétrique, c'est-à-dire, avec deux sources de tension fournissant des tensions opposées, souvent notées \texttt{+VCC}, pour l'alimentation positive, et \texttt{-VCC}, pour l'alimentation négative. 

\Quest A partir de la documentation technique, noter le câblage du composant \texttt{TL081} et les tensions d'alimentation maximales. 

\Manip Réaliser une \textbf{alimentation symétrique +10V / -10V} à partir des alimentations stabilisées disponibles et mettre en place un système de contrôle de ces tensions.

\textit{Vous trouverez des informations concernant les différents instruments dans un fascicule disponible sur l'ensemble des paillasses.}

%%%%%%%%%%%%%%%%%%%%%%%%%%%%%%%%%%
\subsection{Réponse en fréquence de ce montage}

\Manip Réaliser le montage précédent et alimenter le avec l'alimentation symétrique réalisée.

\Manip Tracer le \textbf{diagramme de Bode en gain} de ce système pour des fréquences allant de $100~\operatorname{Hz}$ à $1~\operatorname{MHz}$, à l'aide de mesure réalisée à l'oscilloscope. 

Vous pouvez vous aider des ressources proposées à la fin du document : \hyperref[ressource:RepFreq]{Tracer la réponse en fréquence d'un système linéaire} (Partie \textit{Procédure "classique"}) ainsi .

\Manip Mesurer le \textit{slew-rate} de l'ALI.

%%%%%%%%%%%%%%%%%%%%%%%%%%%%%%%%%%
%%%%%%%%%%%%%%%%%%%%%%%%%%%%%%%%%%
\section{Livrables}


Une \textbf{fiche de manipulation} en ligne (partagée dans le cahier de laboratoire) rappelant :

\begin{itemize}
	\item les protocoles de mesure et de réglage (schémas de mesure, de câblage)
	\item un schéma de câblage de l'alimentation symétrique (et une photo)	
	\item les tableaux de mesures et les courbes obtenues
\end{itemize}

Une \textbf{analyse} du résultat obtenu.



%%%%%%%%%%%%%%%%%%%%%%%%%%%%%%%%%%
\section{Validation}

Les missions 2.1 et 2.2 doivent être validées par un$\cdot$e encadrant$\cdot$e lors de la séance 1 (ou 2). 

Vous présenterez les résultats obtenus ainsi que l'ensemble des documents que vous avez rédigé (cahier de manipulation, courbes...) pour ces deux missions \textbf{à la fin de la mission 2.2}.

%%%%%%%%%%%%%%%%%%%%%%%%%%%%%%%%%%
%%%%%%%%%%%%%%%%%%%%%%%%%%%%%%%%%%
\newpage
\section{Autre ressource}
\label{parslewrate}

\paragraph{A propos du \textit{slew-rate} } Le \textit{slew-rate}  est une limite d'utilisation des ALI en \textbf{régime linéaire}. Cette limitation se traduit par une déformation du signal de sortie comme illustrée par la figure \ref{fig_SlewRate2}.


\begin{figure}[!h]
\centering
\includegraphics[scale=0.6]{images/SlewRate.pdf}
\caption{Illustration de la déformation due à la limitation du slew-rate, pour un montage suiveur (gain unitaire)}  \label{fig_SlewRate2} 
\end{figure}

Le \textit{slew-rate}, $\mathrm{SR}$, est donc quantifié par la pente maximale en $\operatorname{V/\mu s}$ que la tension de sortie peut atteindre.
\begin{equation*}
\mathrm{SR}=\text{max} \left( \frac{dV_\text{s}}{dt} \right )
\end{equation*}

\textbf{Attention aux conditions de mesures, en particulier, toujours s'assurer que le signal n'est pas déformé par le  \textit{slew-rate} lorsqu'on mesure la bande passante !}

%%%%%%%%%%%%%%%%%%%%%%%%%%%%%%%%%%%%%%%%%%%%%%%%%%%%%%%%%%%%%%%%%%%%%%%%%%%%%%%%%%%%%