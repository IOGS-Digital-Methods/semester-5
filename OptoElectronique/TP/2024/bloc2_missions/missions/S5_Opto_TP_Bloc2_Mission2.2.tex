\newpage
\pagestyle{empty}

\begin{minipage}[c]{.25\linewidth}
	\includegraphics[width=4cm]{images/Logo-LEnsE.png}
\end{minipage} \hfill
\begin{minipage}[c]{.4\linewidth}

\begin{center}
\vspace{0.3cm}
{\Large \textsc{Opto-Electronique}}

\medskip

5N-027-SCI \qquad \textbf{\Large TP Bloc 2}

\end{center}
\end{minipage}\hfill

\vspace{0.5cm}

\noindent \rule{\linewidth}{1pt}

{\noindent\Large \rule[-7pt]{0pt}{30pt} \textbf{Mission 2.2} / Mesurer la bande-passante d'un montage linéaire} 

\noindent \rule{\linewidth}{1pt}

\vspace{-0.3cm}

\begin{center}

Durée conseillée : 30 min / Séance 1

\end{center}

%%%%%%%%%%%%%%%%%%%%%%%%%%%%%%%%%%
\section{Objectif de la mission}
\label{mission22}

On se propose de \textbf{mesurer précisément la bande-passante} d'un montage à amplificateur linéaire intégré de type inverseur (voir montage en Mission 2.1).


%%%%%%%%%%%%%%%%%%%%%%%%%%%%%%%%%%
%%%%%%%%%%%%%%%%%%%%%%%%%%%%%%%%%%
%%%%%%%%%%%%%%%%%%%%%%%%%%%%%%%%%%
\section{Ressources}

Vous pouvez utiliser les fiches résumées suivantes : 

\begin{itemize}
	\item \hyperref[fiche:ALIModele]{Fiche : Amplificateur Linéaire Intégré / Modèle}
	\item \hyperref[fiche:AnHaOrdre1]{Fiche : Analyse Harmonique / Ordre 1}
\end{itemize}

%%%%%%%%%%%%%%%%%%%%%%%%%%%%%%%%%%
\subsection{Bande-passante d'un montage}

La \textbf{bande-passante} est un \textbf{paramètre crucial} pour évaluer et concevoir des systèmes électroniques, des filtres, des amplificateurs et des circuits de communication.

Elle est définie comme l'\textbf{intervalle de fréquences} pour lequel le système peut \textbf{transmettre des signaux avec une atténuation minimale}.

D'une autre façon, c'est la gamme de fréquences dans laquelle l'amplitude de la réponse en fréquence du système reste au-dessus d'un certain seuil par rapport à sa valeur maximale.

\medskip

Pour les systèmes linéaires et les filtres, la bande-passante est souvent mesurée entre les points où la puissance du signal de sortie est \textbf{réduite de $3\operatorname{dB}$} par rapport à la puissance du signal de sortie dans la bande-passante.

\medskip

Cette réduction de $3\operatorname{dB}$ peut aussi être interprétée comme une diminution de l'amplitude du signal d'un facteur $\sqrt{2}$ par rapport à l'amplitude du signal de sortie dans la bande-passante.

%%%%%%%%%%%%%%%%%%%%%%%%%%%%%%%%%%
\subsection{Travail à réaliser}

\Manip Faire la mesure précise de la bande-passante du montage.

\Manip Faire la mesure précise de la bande-passante du montage pour un rapport $\frac{R_2}{R_1}$ dix fois plus faible.

%%%%%%%%%%%%%%%%%%%%%%%%%%%%%%%%%%
\section{Livrables}


Une \textbf{fiche de manipulation} en ligne (partagée dans le cahier de laboratoire) rappelant :

\begin{itemize}
	\item le protocole de mesure et de réglage (schémas de mesure, de câblage)
	\item la valeur de la bande-passante d'un montage amplificateur inverseur (voir montage Mission 2.1).
\end{itemize}

Une \textbf{analyse} des résultats obtenus et l'impact du gain sur la bande-passante globale du montage.



%%%%%%%%%%%%%%%%%%%%%%%%%%%%%%%%%%
\subsection{Validation - voir mission 2.1}

%%%%%%%%%%%%%%%%%%%%%%%%%%%%%%%%%%%%%%%%%%%%%%%%%%%%%%%%%%%%%%%%%%%%%%%%%%%%%%%%%%%%%