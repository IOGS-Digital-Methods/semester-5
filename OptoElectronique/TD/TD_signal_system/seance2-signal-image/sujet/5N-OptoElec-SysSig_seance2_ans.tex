%%%%%%%%%%%%%%%%%%%%%%%%%%%%%%%%%%%%%%%%%%
% Engineering problems / LaTeX Template
%		Semester 5
%		Institut d'Optique Graduate School
%%%%%%%%%%%%%%%%%%%%%%%%%%%%%%%%%%%%%%%%%%
%	5N-ONIP-Block1	/ Python for Science
%%%%%%%%%%%%%%%%%%%%%%%%%%%%%%%%%%%%%%%%%%
%
% Created by:
%	Julien VILLEMEJANE - 25/sep/2024
% Modified by:
%	
%
%%%%%%%%%%%%%%%%%%%%%%%%%%%%%%%%%%%%%%%%%%
% Professional Newsletter Template
% LaTeX Template
% Version 1.0 (09/03/14)
%
% Created by:
% Bob Kerstetter (https://www.tug.org/texshowcase/) and extensively modified by:
% Vel (vel@latextemplates.com)
% 
% This template has been downloaded from:
% http://www.LaTeXTemplates.com
%
% License:
% CC BY-NC-SA 3.0 (http://creativecommons.org/licenses/by-nc-sa/3.0/)
%
%%%%%%%%%%%%%%%%%%%%%%%%%%%%%%%%%%%%%%%%%

\documentclass[10pt]{article} % The default font size is 10pt; 11pt and 12pt are alternatives

\input{../../../../../_assets/latex/5N_ONIP_structure.tex} % Include the document which specifies all packages and structural customizations for this template
\usepackage{amsmath}

%----------------------------------------------------------------------------------------
%	DOCUMENT INFORMATIONS
%----------------------------------------------------------------------------------------
\def\module{Opto-Electronique}
\def\submodule{OptoElec}
\def\moduleSmall{5N-027-SCI / OE}
\def\year{2024-2025}
\def\problem{TD Systèmes et Signaux}
\def\problemName{OptoElec \& ONIP-1 / TD Systèmes et Signaux}

\def\validation{}

\def\scheduleCM{0}
\def\scheduleTD{0}
\def\scheduleTDcomputer{2}
\def\scheduleTP{0}

\def\workingTeam{Par binôme}

\def\workingSpecial{}

\def\keywords{Systèmes asservis;ALI;FFT}


\begin{document}
%----------------------------------------------------------------------------------------
%	HEADER IMAGE
%----------------------------------------------------------------------------------------

\begin{figure}[H]
\centering\includegraphics[width=0.3\linewidth]{../../../../../_assets/latex/logo_iogs.png}
\end{figure}

%----------------------------------------------------------------------------------------
%	MAIN BODY - FIRST PAGE
%----------------------------------------------------------------------------------------
%
\begin{minipage}[t]{1.0\linewidth} % Mini page taking up 65% of the actual page

\hypertarget{context}{\heading{\huge \problemName}{6pt}} % \hypertarget provides a label to reference using \hyperlink{label}{link text}

%-----------------------------------
\centerline {\rule{.70\linewidth}{.25pt}} % Horizontal line
\begin{center}
\textsc{\large Séance 2 / Signaux, images et FFT}
\end{center}
\centerline {\rule{.70\linewidth}{.25pt}} % Horizontal line
%-----------------------------------

\medskip

\begin{mdframed}[style=sidebar,frametitle={}]
\large
Un tutoriel sur la FFT est proposé sur la page suivante, section \textbf{\large FFT with Python} :

\begin{center}
\Large
\href{https://iogs-lense-training.github.io/python-for-science/}{https://iogs-lense-training.github.io/python-for-science/}
\end{center}

\end{mdframed}

%-----------------------------------
\textbf{Exercice 1 / FFT sur des signaux 1D}

\begin{enumerate}
	\item Définir une fonction qui génère un signal sinusoïdal d'une amplitude donnée et d'une fréquence donnée. Le vecteur temps devra également être passé en argument.
	\item Générer 2 signaux sinusoïdaux :
	\begin{itemize}
		\item le premier $sine\_a$ de fréquence $200\operatorname{Hz}$ et d'amplitude $1$
		\item le second $sine\_b$ de fréquence $287\operatorname{Hz}$ et d'amplitude $2$
	\end{itemize}
	\item Tracer ces deux signaux sur un même graphique.

	On se propose d'étudier un signal $sine\_c$ correspondant à la somme de ces deux signaux : $sine\_c = sine\_a + sine\_b$
	
	\item Tracer le signal $sine\_c$ sur un graphique.
	
	\medskip
	
	\item Calculer la FFT de ce signal et tracer cette réponse en fréquence sur un nouveau graphique, sans spécifier l'axe des fréquences. Que pouvez-vous conclure ?
	
	\medskip
	
	\item Construire l'axe des fréquences à l'aide de la fonction \textsl{numpy.fft.fftfreq}.
	\item Afficher la FFT du signal précédent en se basant cette fois-ci sur l'axe des fréquences générés à l'aide de \textsl{numpy.fft.fftfreq}.
\end{enumerate}



\centerline {\rule{.70\linewidth}{.25pt}} % Horizontal line
%-----------------------------------
\textbf{Exercice 2 / FFT sur des images}

On se propose d'étudier le script \textsl{image\_filter.py} (associé à l'image \textsc{test\_image.png}), disponible sur le site du LEnsE, rubrique Année / Première Année / Outils Numériques / TD Systèmes et Signaux.

\begin{enumerate}
	\item Tester ce script.
	\item Modifier la valeur du rayon (\textit{radius}) associée à la fonction \textsl{circular\_mask.py} et relancer le script.
	\item Expliquer les différentes étapes et l'impact sur l'image finale.
\end{enumerate}


\centerline {\rule{.70\linewidth}{.25pt}} % Horizontal line
%-----------------------------------
\textbf{Exercice 3 / FFT en 2D}

Le script \textsl{trame\_generator.py}, disponible sur le site du LEnsE, rubrique Année / Première Année / Outils Numériques / TD Systèmes et Signaux, contient une fonction permettant de générer des trames sinusoïdales en 2D.

\begin{enumerate}
	\item A l'aide de cette fonction, générer une trame sinusoïdale de taille $400\operatorname{pixels}$ par $300\operatorname{pixels}$ avec un angle de $63^\circ$ et un pas de $30\operatorname{pixels}$.
	\item Afficher l'image.
	\item Tracer et afficher la FFT de cette image (attention au \textit{shift}).
	\item Tester pour des pas et des angles différents.
\end{enumerate}

\end{minipage}

\end{document} 