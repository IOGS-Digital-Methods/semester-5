\documentclass[a4paper]{book}%
\input{../../../../_assets/latex/5N_ONIP_structure.tex}
\usepackage{amsmath}

%----------------------------------------------------------------------------------------
%	DOCUMENT
%----------------------------------------------------------------------------------------
\begin{document}

%----------------------------------------------------------------------------------------
% SUJET 1
%----------------------------------------------------------------------------------------

	\Large 
\begin{tabular}{c}
\includegraphics[width=5cm]{logoLEnsE.png}\\
Cycle ingénieur 1A
\end{tabular}
\hfill
\begin{tabular}{c}
Travaux Pratiques Opto-Électronique \\
Semestre 5 \\
\end{tabular}\\
\normalsize 

\bigskip

\begin{mdframed}[style=aavbox,frametitle={Test individuel}]
	
L'objectif principal de ce test est l'auto-évaluation de l'acquisition individuelle des savoirs et savoir-faire.

Vous avez 2 heures pour traiter les deux parties proposées par la suite, en \textbf{autonomie} :
\begin{itemize}
	\item concevoir et réaliser un circuit sur une plaquette de prototypage
	\item proposer un protocole expérimental
	\item mettre en \oe{}uvre un protocole de mesure adapté
	\item analyser les mesures réalisées
\end{itemize}

Une grille d'auto-évaluation est fournie au verso de cette page. 

\textbf{Vous avez accès à toutes les ressources documentaires.}
\end{mdframed}	

\medskip	

	\noindent \hrulefill
	
	\section*{Partie A / Etude d'un dipôle}
	
	On souhaite caractériser le dipôle mis à votre disposition. 

	\begin{enumerate}
		\item Tracer la caractéristique statique de ce dipôle.
		\item Relever les paramètres importants d'utilisation.
	\end{enumerate}

	\noindent \hrulefill

\textbf{ATTENTION} : Ne dépassez pas un courant de $20\operatorname{mA}$ et une tension inverse de $4\operatorname{V}$ pour le dipôle étudié.

	\noindent \hrulefill
	
	\section*{Partie B / Caractérisation d'un système}

	
	On se propose d'étudier le montage suivant : 
	
	\begin{center}
\begin{circuitikz} 
	\node [op amp, fill=blue!10!white](A1) at (0,0){\texttt{ALI1}};
	\draw (A1.-) to[short] ++(-.5,0) coordinate(A) to[short] ++(0,1.5) coordinate(B) to[R, l^=$R_2$] (B -| A1.out) coordinate(RR) to[short, -*] (A1.out);
	\draw (A1.-) to[short,-*] ++(-.5,0) coordinate(AA) to[C, l_=$C$] ++(-2,0) to[R, l_=$R_1$] ++(-2.5,0) coordinate(BB) to [short,l_=${V_e}$, -o] ++(-1,0) coordinate(CC);

	\draw (A1.+) to[short] ++(-.5,0) coordinate(AA2) node[cground]{};

	\draw (A1.out) to [short,l=${V_S}$, -o] ++(1,0) coordinate(D);
	
\end{circuitikz}
\end{center}		

\bigskip

Avec $R_1 = 33\operatorname{k\Omega}$, $R_2 = 100\operatorname{k\Omega}$ et $C = 10\operatorname{nF}$. L'ALI sera alimenté avec une alimentation symétrique de +/- $12\operatorname{V}$.

\bigskip
	
	\begin{enumerate}
		\item Faire une étude asymptotique en fréquence de ce montage.
		\item Réaliser le montage proposé ci-dessus.
		\item Tracer la \textbf{réponse en fréquence} de ce système et évaluer les \textbf{principales caractéristiques} (gain, bande-passante, ordre, déphasage pour des valeurs pertinentes de fréquence...).
	\end{enumerate}	
	
	
\includepdf[pages=-,landscape=true]{../S5_Optoelectronique_2024_AutoEval_Indiv.pdf}	


\cleardoublepage
%----------------------------------------------------------------------------------------
% SUJET 2
%----------------------------------------------------------------------------------------

	\Large 
\begin{tabular}{c}
\includegraphics[width=5cm]{logoLEnsE.png}\\
Cycle ingénieur 1A
\end{tabular}
\hfill
\begin{tabular}{c}
Travaux Pratiques Opto-Électronique \\
Semestre 5 \\
\end{tabular}\\
\normalsize 

\bigskip

\begin{mdframed}[style=aavbox,frametitle={Test individuel}]
	
L'objectif principal de ce test est l'auto-évaluation de l'acquisition individuelle des savoirs et savoir-faire.

Vous avez 2 heures pour traiter les deux parties proposées par la suite, en \textbf{autonomie} :
\begin{itemize}
	\item concevoir et réaliser un circuit sur une plaquette de prototypage
	\item proposer un protocole expérimental
	\item mettre en \oe{}uvre un protocole de mesure adapté
	\item analyser les mesures réalisées
\end{itemize}

Une grille d'auto-évaluation est fournie au verso de cette page. 

\textbf{Vous avez accès à toutes les ressources documentaires.}
\end{mdframed}	

\medskip 

	\noindent \hrulefill
	
	\section*{Partie A / Etude d'un dipôle}
	
	On souhaite caractériser le dipôle mis à votre disposition. 

	\begin{enumerate}
		\item Tracer la caractéristique statique de ce dipôle.
		\item Relever les paramètres importants d'utilisation.
	\end{enumerate}

	\noindent \hrulefill

\textbf{ATTENTION} : Ne dépassez pas un courant de $20\operatorname{mA}$ et une tension inverse de $4\operatorname{V}$ pour le dipôle étudié.

	\noindent \hrulefill
	
	\section*{Partie B / Caractérisation d'un système}
	
	On se propose d'étudier le montage suivant : 
	
	\begin{center}
\begin{circuitikz} 
	\node [op amp, fill=blue!10!white](A1) at (0,0){\texttt{ALI1}};
	\draw (A1.-) to[short] ++(-.5,0) coordinate(A) to[short] ++(0,1.5) coordinate(B) to[R, l^=$R_2$] (B -| A1.out) coordinate(RR) to[short, -*] (A1.out);
	\draw (A1.-) to[short,-*] ++(-.5,0) coordinate(AA) to[R, l_=$R_1$] ++(-2.5,0) coordinate(BB) to [short,l_=${V_e}$, -o] ++(-1,0) coordinate(CC);

	\draw (B) to[short,*-] ++ (0, 1.5) coordinate(RC) to[C, l=$C$] (RC -| A1.out) to[short,-*] (RR);
	\draw (A1.+) to[short] ++(-.5,0) coordinate(AA2) node[cground]{};

	\draw (A1.out) to [short,l=${V_S}$, -o] ++(1,0) coordinate(D);
	
\end{circuitikz}
\end{center}		

\bigskip

Avec $R_1 = 2.2\operatorname{k\Omega}$, $R_2 = 10\operatorname{k\Omega}$ et $C = 100\operatorname{nF}$. L'ALI sera alimenté avec une alimentation symétrique de +/- $12\operatorname{V}$.

\bigskip
	
	\begin{enumerate}
		\item Faire une étude asymptotique en fréquence de ce montage.
		\item Réaliser le montage proposé ci-dessus.
		\item Tracer la \textbf{réponse en fréquence} de ce système et évaluer les \textbf{principales caractéristiques} (gain, bande-passante, ordre, déphasage pour des valeurs pertinentes de fréquence...).
	\end{enumerate}	

	
\includepdf[pages=-,landscape=true]{../S5_Optoelectronique_2024_AutoEval_Indiv.pdf}

\cleardoublepage
%----------------------------------------------------------------------------------------
% SUJET 3
%----------------------------------------------------------------------------------------

	\Large 
\begin{tabular}{c}
\includegraphics[width=5cm]{logoLEnsE.png}\\
Cycle ingénieur 1A
\end{tabular}
\hfill
\begin{tabular}{c}
Travaux Pratiques Opto-Électronique \\
Semestre 5 \\
\end{tabular}\\
\normalsize 

\bigskip

\begin{mdframed}[style=aavbox,frametitle={Test individuel}]
	
L'objectif principal de ce test est l'auto-évaluation de l'acquisition individuelle des savoirs et savoir-faire.

Vous avez 2 heures pour traiter les deux parties proposées par la suite, en \textbf{autonomie} :
\begin{itemize}
	\item concevoir et réaliser un circuit sur une plaquette de prototypage
	\item proposer un protocole expérimental
	\item mettre en \oe{}uvre un protocole de mesure adapté
	\item analyser les mesures réalisées
\end{itemize}

Une grille d'auto-évaluation est fournie au verso de cette page. 

\textbf{Vous avez accès à toutes les ressources documentaires.}
\end{mdframed}	

\medskip 
	

	\noindent \hrulefill
	
	\section*{Partie A / Etude d'un dipôle}
	
	On souhaite caractériser le dipôle mis à votre disposition.  

	\begin{enumerate}
		\item Tracer la caractéristique statique de ce dipôle.
		\item Relever les paramètres importants d'utilisation.
	\end{enumerate}

	\noindent \hrulefill

\textbf{ATTENTION} : Ne dépassez pas un courant de $20\operatorname{mA}$ et une tension inverse de $4\operatorname{V}$ pour le dipôle étudié.

	\noindent \hrulefill
	
	\section*{Partie B / Caractérisation d'un système}
	
	On se propose d'étudier le montage suivant : 
	
	
	\begin{center}
\begin{circuitikz} 
	\node [op amp, fill=blue!10!white](A1) at (0,0){\texttt{ALI1}};
	\draw (A1.-) to[short] ++(-.5,0) coordinate(A) to[short] ++(0,1.5) coordinate(B) to[R, l^=$R_2$] (B -| A1.out) coordinate(RR) to[short, -*] (A1.out);
	\draw (A1.-) to[short,-*] ++(-.5,0) coordinate(AA) to[R, l_=$R_1$] ++(-2.5,0) coordinate(BB) to [short,l_=${V_e}$, -o] ++(-1,0) coordinate(CC);

	\draw (A1.+) to[short] ++(-.5,0) coordinate(AA2) node[cground]{};

	\draw (A1.out) to [short,l=${V_S}$, -o] ++(1,0) coordinate(D);
	
\end{circuitikz}
\end{center}	

\bigskip

Avec $R_1 = 1\operatorname{k\Omega}$ et $R_2 = 33\operatorname{k\Omega}$. L'ALI sera alimenté avec une alimentation symétrique de +/- $12\operatorname{V}$.
	
\bigskip

	\begin{enumerate}
		\item Faire une étude asymptotique en fréquence de ce montage.
		\item Réaliser le montage proposé ci-dessus.
		\item Tracer la \textbf{réponse en fréquence} de ce système et évaluer les \textbf{principales caractéristiques} (gain, bande-passante, ordre, déphasage pour des valeurs pertinentes de fréquence...).
	\end{enumerate}		
		
	
	
\includepdf[pages=-,landscape=true]{../S5_Optoelectronique_2024_AutoEval_Indiv.pdf}

\cleardoublepage
%----------------------------------------------------------------------------------------
% SUJET 4
%----------------------------------------------------------------------------------------

	\Large 
\begin{tabular}{c}
\includegraphics[width=5cm]{logoLEnsE.png}\\
Cycle ingénieur 1A
\end{tabular}
\hfill
\begin{tabular}{c}
Travaux Pratiques Opto-Électronique \\
Semestre 5 \\
\end{tabular}\\
\normalsize 

\bigskip

\begin{mdframed}[style=aavbox,frametitle={Test individuel}]
	
L'objectif principal de ce test est l'auto-évaluation de l'acquisition individuelle des savoirs et savoir-faire.

Vous avez 2 heures pour traiter les deux parties proposées par la suite, en \textbf{autonomie} :
\begin{itemize}
	\item concevoir et réaliser un circuit sur une plaquette de prototypage
	\item proposer un protocole expérimental
	\item mettre en \oe{}uvre un protocole de mesure adapté
	\item analyser les mesures réalisées
\end{itemize}

Une grille d'auto-évaluation est fournie au verso de cette page. 

\textbf{Vous avez accès à toutes les ressources documentaires.}
\end{mdframed}	

\medskip 

	\section*{Partie A / Etude d'un dipôle}
	
	On souhaite caractériser le dipôle mis à votre disposition. 

	\begin{enumerate}
		\item Tracer la caractéristique statique de ce dipôle.
		\item Relever les paramètres importants d'utilisation.
	\end{enumerate}

	\noindent \hrulefill

\textbf{ATTENTION} : Ne dépassez pas un courant de $20\operatorname{mA}$ et une tension inverse de $4\operatorname{V}$ pour le dipôle étudié.

	\noindent \hrulefill
	
	\section*{Partie B / Caractérisation d'un système}

On se propose d'étudier le filtre 2 de la maquette proposée ($IN2$ et $OUT2$).

Il s'agit d'un \textbf{filtre actif} qui nécessite une \textbf{alimentation symétrique} : 
\begin{itemize}
	\item NOIRE : masse
	\item ROUGE : $+V_{CC}$
	\item BLEU : $-V_{CC}$
\end{itemize}

\medskip

\begin{enumerate}
	\item Réaliser l'alimentation symétrique avec $V_{CC} = 12V$.
	\item Alimenter ensuite la maquette.	
	\item Tracer la \textbf{réponse en fréquence} de ce système et évaluer les \textbf{principales caractéristiques} (gain, bande-passante, ordre, déphasage pour des valeurs pertinentes de fréquence...).
\end{enumerate}	

\includepdf[pages=-,landscape=true]{../S5_Optoelectronique_2024_AutoEval_Indiv.pdf}

\end{document}