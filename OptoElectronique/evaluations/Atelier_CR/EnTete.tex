\Large 
\begin{tabular}{c}
\includegraphics[width=5cm]{logoLEnsE.png}
\end{tabular}
\hfill
\begin{tabular}{c}
Travaux Pratiques CéTI \\
Semestre 5 \\
\end{tabular}\\
\normalsize 

\bigskip

\begin{mdframed}[style=aavbox,frametitle={Travail de synthèse}]

L'objectif de l'atelier est de rédiger en temps limité un compte-rendu du travail expérimental accompli dans le cadre du bloc 1 ou du bloc 2. Ce document synthétique et riche de preuves (basées sur vos résultats expérimentaux) a pour objectif de revendiquer le fait que vous êtes capable :

\begin{enumerate}
	\item soit de \textbf{caractériser un dipôle électronique} (linéaire ou non-linéaire) statiquement et en déduire ses zones de fonctionnement, 
	\item soit de \textbf{caractériser un système linéaire} dans les domaines temporel et fréquentiel.
\end{enumerate}

Le livrable de synthèse est évalué en cours de séance par l'encadrant·e. 
\end{mdframed}	

\medskip