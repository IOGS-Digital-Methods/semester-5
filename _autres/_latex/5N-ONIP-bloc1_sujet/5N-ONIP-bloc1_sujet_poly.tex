%%%%%%%%%%%%%%%%%%%%%%%%%%%%%%%%%%%%%%%%%%
% Engineering problems / LaTeX Template
%		Semester 5
%		Institut d'Optique Graduate School
%%%%%%%%%%%%%%%%%%%%%%%%%%%%%%%%%%%%%%%%%%
%	5N-ONIP-Block1	/ Python et calcul scientifique
%%%%%%%%%%%%%%%%%%%%%%%%%%%%%%%%%%%%%%%%%%
%
% Created by:
%	Julien VILLEMEJANE - 05/may/2023
% Modified by:
%	
%
%%%%%%%%%%%%%%%%%%%%%%%%%%%%%%%%%%%%%%%%%%
% Professional Newsletter Template
% LaTeX Template
% Version 1.0 (09/03/14)
%
% Created by:
% Bob Kerstetter (https://www.tug.org/texshowcase/) and extensively modified by:
% Vel (vel@latextemplates.com)
% 
% This template has been downloaded from:
% http://www.LaTeXTemplates.com
%
% License:
% CC BY-NC-SA 3.0 (http://creativecommons.org/licenses/by-nc-sa/3.0/)
%
%%%%%%%%%%%%%%%%%%%%%%%%%%%%%%%%%%%%%%%%%

\documentclass[10pt]{article} % The default font size is 10pt; 11pt and 12pt are alternatives

\input{../../_latex_assets/5N_ONIP_structure.tex} % Include the document which specifies all packages and structural customizations for this template

\usepackage{pdfpages}

\def\dirName{5N-ONIP-bloc1_sujet}

\begin{document}


%----------------------------------------------------------------------------------------
%	DOCUMENT INFORMATIONS
%----------------------------------------------------------------------------------------
\def\module{ONIP}
\def\submodule{Outils Numériques}
\def\moduleSmall{5N-028-Phy}
\def\year{2023-2024}
\def\problemName{Outils Numériques pour l'Ingénieur$\cdot$e en Physique}

\def\scheduleCM{0.5}
\def\scheduleTD{0}
\def\scheduleTDcomputer{3 x 4}
\def\scheduleTP{0}

\def\workingTeam{Seul ou par équipe de 2}

\def\workingSpecial{}

\def\keywords{Physique; Outils Numériques; Méthodes Numériques; Python;}


%----------------------------------------------------------------------------------------
%	HEADER IMAGE
%----------------------------------------------------------------------------------------

\begin{figure}[H]
\centering\includegraphics[width=0.5\linewidth]{../../_latex_assets/logo_iogs.png}
\end{figure}

%----------------------------------------------------------------------------------------
%	SIDEBAR - FIRST PAGE
%----------------------------------------------------------------------------------------

\begin{minipage}[t]{.35\linewidth} % Mini page taking up 30% of the actual page
\begin{mdframed}[style=sidebar,frametitle={\module}] % Sidebar box

%-----------------------------------------------------------
%	DOCUMENT DESCRIPTION
\begin{center}
\bigskip

\textit{\large \centering \year}
\end{center}


\centerline {\rule{.70\linewidth}{.25pt}} % Horizontal line

\begin{center}
	\textit{\large \moduleSmall}
\end{center}

\centerline {\rule{.70\linewidth}{.25pt}} % Horizontal line

%-----------------------------------------------------------

\textbf{Objectif principal}

Construire une boite à outils de méthodes numériques pour de futur.es ingénieur.es en physique

\centerline {\rule{.70\linewidth}{.25pt}} % Horizontal line

%-----------------------------------------------------------

\textbf{Mots clefs}

\keywords

\centerline {\rule{.70\linewidth}{.25pt}} % Horizontal line

%-----------------------------------------------------------

\textbf{Sessions}

\begin{itemize}
\item[\textbf{\scheduleCM}] Cours(s) - 1h30
\item[\textbf{\scheduleTD}] TD(s) - 1h30
\item[\textbf{\scheduleTDcomputer}] TD(s) Machine - 2h00
\item[\textbf{\scheduleTP}] TP(s) - 4h30
\end{itemize}

\centerline {\rule{.70\linewidth}{.25pt}} % Horizontal line

{\large Travail}

\textbf{\workingTeam}

\textbf{\workingSpecial}


%-----------------------------------------------------------

\end{mdframed}


\centering
\begin{minipage}[t]{.95\linewidth}
\textbf{Institut d'Optique}\\
Graduate School, \textit{France}\\
\href{https://www.institutoptique.fr}{https://www.institutoptique.fr}

\medskip
\textbf{GitHub - Digital Methods}

\href{https://github.com/IOGS-Digital-Methods}{https://github.com/IOGS-Digital-Methods}

\end{minipage}

\end{minipage}\hfill % End the sidebar mini page 
%
%----------------------------------------------------------------------------------------
%	MAIN BODY - FIRST PAGE
%----------------------------------------------------------------------------------------
%
\begin{minipage}[t]{.60\linewidth} % Mini page taking up 66% of the actual page

\hypertarget{context}{\heading{\huge \problemName}{6pt}} % \hypertarget provides a label to reference using \hyperlink{label}{link text}

\centerline {\rule{.70\linewidth}{.25pt}} % Horizontal line

\section*{Responsables pédagogiques}

\begin{tabular}{r l l}
	\textsc{Sylvie LEBRUN} & sylvie.lebrun@institutoptique.fr \\
	\textsc{Julien VILLEMEJANE}& julien.villemejane@institutoptique.fr \\ 
\end{tabular}

%% Short introduction 
\section*{Déroulement du module}

Ce module est décomposé en \textbf{3 thèmes} de \textit{4 séances chacun} :
\begin{itemize}
	\item Méthodes numériques
	\item Traitement de données 2D 
	\item Traitement de données 1D 
\end{itemize}


\section*{Déroulement d'un thème}

Chaque thème sera découpé de la façon suivante :

\begin{description}
	\item[Séance 1] Appropriation de la problématique
	\item[Séance 2] Mise en oeuvre numérique
	\item[Séance 3] Mise en forme des résultats / Evaluation
	\item[Séance 4] Synthèse / Evaluation
\end{description}

%%

\bigskip

%----------------------------------------------------------------------------------------
%	IN-TEXT BOX / Intended learning outcomes
%----------------------------------------------------------------------------------------

\begin{mdframed}[style=aavbox,frametitle={Evaluation du module}]
A la fin de chaque thématique, votre travail sera évalué selon la grille proposée en page 2 de ce document par l'un ou l'une des encadrant.es. 

Les thèmes 2 et 3 feront l'objet d'une notation à hauteur de 50\% pour chacun des blocs.

\textit{Vous pouvez vous servir d'une partie des grilles pour vous auto-évaluer.}
\end{mdframed}



\end{minipage} % End the main body - first page mini page

\bigskip

\begin{center}
	\Large Retrouvez l'ensemble des documents pédagogiques sur
	
	\href{http://lense.institutoptique.fr/ONIP/}{http://lense.institutoptique.fr/ONIP/}
	
	\medskip
	
	\large	
	
	ou sur le dépôt GitHub suivant :
	
	\href{https://github.com/IOGS-Digital-Methods/semester-5}{https://github.com/IOGS-Digital-Methods/semester-5}
\end{center}

\includepdf[pages=-,angle=90]{docs/Grilles_Eval_ONIP}
\includepdf[pages=-]{docs/onip_b1_deroulement}
\includepdf[pages=-]{5N-ONIP-bloc1_sujet_fr}

%----------------------------------------------------------------------------------------

\end{document} 