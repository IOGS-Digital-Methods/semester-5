%%%%%%%%%%%%%%%%%%%%%%%%%%%%%%%%%%%%%%%%%%
% Engineering problems / LaTeX Template
%		Semester 5
%		Institut d'Optique Graduate School
%%%%%%%%%%%%%%%%%%%%%%%%%%%%%%%%%%%%%%%%%%
%	EngPb_01	/ Global Physicist	
%%%%%%%%%%%%%%%%%%%%%%%%%%%%%%%%%%%%%%%%%%
%
% Created by:
%	Julien VILLEMEJANE - 08/dec/2022
% Modified by:
%	
%
%%%%%%%%%%%%%%%%%%%%%%%%%%%%%%%%%%%%%%%%%%
% Professional Newsletter Template
% LaTeX Template
% Version 1.0 (09/03/14)
%
% Created by:
% Bob Kerstetter (https://www.tug.org/texshowcase/) and extensively modified by:
% Vel (vel@latextemplates.com)
% 
% This template has been downloaded from:
% http://www.LaTeXTemplates.com
%
% License:
% CC BY-NC-SA 3.0 (http://creativecommons.org/licenses/by-nc-sa/3.0/)
%
%%%%%%%%%%%%%%%%%%%%%%%%%%%%%%%%%%%%%%%%%

\documentclass[10pt]{article} % The default font size is 10pt; 11pt and 12pt are alternatives

\input{assets/structure.tex} % Include the document which specifies all packages and structural customizations for this template

\begin{document}


%----------------------------------------------------------------------------------------
%	DOCUMENT INFORMATIONS
%----------------------------------------------------------------------------------------
\def\module{Data Processing\\ Problems}
\def\submodule{Numerical Tools}
\def\moduleSmall{5N-027-SCI / DPP\_NT}
\def\year{2023-2024}
\def\problem{DPP\_Pb\_S5\_01}
\def\problemName{Displaying scientific data}

\def\validation{10\%}

\def\scheduleCM{0}
\def\scheduleTD{0}
\def\scheduleTDcomputer{1}
\def\scheduleTP{0}

\def\workingTeam{team of 2}


%----------------------------------------------------------------------------------------
%	HEADER IMAGE
%----------------------------------------------------------------------------------------

\begin{figure}[H]
\centering\includegraphics[width=0.5\linewidth]{assets/logo.png}
\end{figure}

%----------------------------------------------------------------------------------------
%	SIDEBAR - FIRST PAGE
%----------------------------------------------------------------------------------------

\begin{minipage}[t]{.35\linewidth} % Mini page taking up 30% of the actual page
\begin{mdframed}[style=sidebar,frametitle={}] % Sidebar box

%-----------------------------------------------------------
%	DOCUMENT DESCRIPTION
\begin{center}
\textbf{\Large \module}

\textit{\centering \year}
\end{center}


\centerline {\rule{.70\linewidth}{.25pt}} % Horizontal line

\begin{center}
	\textit{\large \moduleSmall}
\end{center}

\centerline {\rule{.70\linewidth}{.25pt}} % Horizontal line

\begin{center}
	\textbf{\large \problem} ( \validation )
\end{center}

\centerline {\rule{.70\linewidth}{.25pt}} % Horizontal line

%-----------------------------------------------------------

\begin{itemize}
\item \hyperlink{context}{Context} % These link to their appropriate sections in the newsletter
\item \hyperlink{ilos}{Intended Learning Outcomes}
\item \hyperlink{deliverables}{Deliverables}
\item \hyperlink{coveredconcepts}{Covered Concepts}
\item \hyperlink{stepbystep}{Step by step}
\item \hyperlink{ressources}{Ressources}
\end{itemize}


\centerline {\rule{.70\linewidth}{.25pt}} % Horizontal line

%-----------------------------------------------------------

\textbf{Keywords}

Data Processing ; 

\centerline {\rule{.70\linewidth}{.25pt}} % Horizontal line

%-----------------------------------------------------------

\textbf{Sessions}

\begin{itemize}
\item[\textbf{\scheduleCM}] Course(s) - 1h30
\item[\textbf{\scheduleTD}] Directed Session(s) - 1h30
\item[\textbf{\scheduleTDcomputer}] Computer Session(s) - 2h00
\item[\textbf{\scheduleTP}] Practical Session(s) - 4h30
\end{itemize}

\centerline {\rule{.70\linewidth}{.25pt}} % Horizontal line

{\large Working by \textbf{\workingTeam}}


%-----------------------------------------------------------

\end{mdframed}


\centering
\begin{minipage}[t]{.95\linewidth}
\textbf{Institut d'Optique}\\
Graduate School, \textit{France}\\
\href{https://www.institutoptique.fr}{https://www.institutoptique.fr}
\end{minipage}

\end{minipage}\hfill % End the sidebar mini page 
%
%----------------------------------------------------------------------------------------
%	MAIN BODY - FIRST PAGE
%----------------------------------------------------------------------------------------
%
\begin{minipage}[t]{.60\linewidth} % Mini page taking up 66% of the actual page

\hypertarget{context}{\heading{\huge \problemName}{6pt}} % \hypertarget provides a label to reference using \hyperlink{label}{link text}

%% Short introduction 

Producing \textbf{scientific reports} with \textbf{smooth and clear graphics} is a large part of the life of an engineer or a researcher.

In this session, you will have to : 
\begin{itemize}
	\item generate discrete sinewaves at different frequencies - from a period (or frequency), a number of periods and a sampling frequency 
	\item plot those signals depending on a time vector (or different time vectors)
	\item fill the legend, the axis labels of the graphics	
\end{itemize}

%%

\bigskip

%----------------------------------------------------------------------------------------
%	IN-TEXT BOX / Intended learning outcomes
%----------------------------------------------------------------------------------------

\begin{mdframed}[style=intextbox,frametitle={}] % Sidebar box


\hypertarget{ilos}{\heading{Intended Learning Outcomes}{0pt}} % \hypertarget provides a label to reference using \hyperlink{label}{link text}

By solving this problem, students will be able to :
\begin{enumerate}
\item \textbf{translate a mathematical function} into a programming function
\item \textbf{produce scientific smooth and clear graphics} from a mathematical function - including titles, legends and axis labels - with \textit{Matploblib.pyplot} Python library
\end{enumerate}

\end{mdframed}
\medskip

%----------------------------------------------------------------------------------------
%	IN-TEXT BOX / Deliverables
%----------------------------------------------------------------------------------------

\begin{mdframed}[style=intextbox,frametitle={}] % Sidebar box


\hypertarget{deliverables}{\heading{Deliverables}{0pt}} % \hypertarget provides a label to reference using \hyperlink{label}{link text}

At the end of the session, you have to produce :
\begin{enumerate}
\item \textbf{commented functions} to generate specific signals (such as a sinewave) from a time vector or other parameters (sampling frequency, frequency of the signal, number of periods)
\item graphics with titles, legends and axis labels
\end{enumerate}

\end{mdframed}


%\begin{wrapfigure}[7]{l}[0pt]{0pt} % In-line figure with text wrapping around it
%\includegraphics[width=0.3\textwidth]{engPb_S5_01/placeholder.jpg}
%\end{wrapfigure}

\end{minipage} % End the main body - first page mini page

%----------------------------------------------------------------------------------------
%	MAIN BODY - SECOND PAGE
%----------------------------------------------------------------------------------------

\begin{minipage}[t]{.66\linewidth} % Mini page taking up 66% of the actual page



%-----------------------------------------------------------

\hypertarget{coveredconcepts}{\heading{Covered Concepts}{6pt}} % \hypertarget provides a label to reference using \hyperlink{label}{link text}

This session refers to :

\begin{itemize}
	\item mathematical trigonometric functions
	\item vectors for data storage
	\item Python libraries for data displaying
\end{itemize}


\centerline {\rule{.70\linewidth}{.25pt}} % Horizontal line

%-----------------------------------------------------------

\hypertarget{stepbystep}{\heading{Step by step}{6pt}} % \hypertarget provides a label to reference using \hyperlink{label}{link text}

During this session, you can follow the steps below :

\begin{enumerate}
	\item open a new Jupyter Notebook script (or basic python script)
	\item 
\end{enumerate}

\centerline {\rule{.70\linewidth}{.25pt}} % Horizontal line

%-----------------------------------------------------------

\hypertarget{ressources}{\heading{Ressources}{6pt}} % \hypertarget provides a label to reference using \hyperlink{label}{link text}

This session is based on Python programming language.

You can use the \textbf{JupyterHub@Paris-Saclay} environment - 
\href{https://jupyterhub.ijclab.in2p3.fr/}{https://jupyterhub.ijclab.in2p3.fr/} or \textbf{Spyder} development environment from \textit{Anaconda}.

You will find tutorials on Python (and basics library as Numpy, Matplotlib or Scipy) at : \href{http://lense.institutoptique.fr/python/}{http://lense.institutoptique.fr/python/}, we suggest to you to read the following ones :

\begin{itemize}
	\item How to create a function properly (with documented comments)
	\item How to create a vector (linear or logarithm evolution)
	\item How to plot data from vectors 
\end{itemize}


%----------------------------------------------------------------------------------------

\end{minipage}\hfill % End of the main body - second page mini page
\begin{minipage}[t]{.30\linewidth} % Mini page taking up 30% of the actual page

%----------------------------------------------------------------------------------------
%	SIDEBAR - SECOND PAGE
%----------------------------------------------------------------------------------------

\begin{mdframed}[style=sidebar,frametitle={}] % Sidebar box

\heading{Numerical Tools}{0pt}

Functions and libraries to use :

\begin{itemize}
	\item \textbf{Numpy} for mathematical functions and vectors generation :
	\begin{itemize}
		\item \textbf{linspace}
		\item \textbf{logspace}	
	\end{itemize}
	\item \textbf{Matplotlib} for plotting data
	\begin{itemize}
		\item \textbf{plotly} sublibrary
		\item \textbf{figure}
		\item \textbf{plot}
		\item \textbf{legend}	
		\item \textbf{xlabel, ylabel}	
		\item \textbf{title}
		\item \textbf{show}	
	\end{itemize}	
	\item \textbf{Scipy} for scientific functions
	\begin{itemize}
		\item \textbf{fftpack} sublibrary
		\item \textbf{fft}
	\end{itemize}
\end{itemize}


\centerline {\rule{.70\linewidth}{.25pt}} % Horizontal line


Advanced tools :

\begin{itemize}
	\item \textbf{rcParams} from MatPlotLib.pyplot	
	\item \textbf{Scipy} for scientific functions
	\begin{itemize}
		\item \textbf{special} sublibrary
		\item \textbf{jv} Bessel function 
	\end{itemize}
\end{itemize}

\end{mdframed}\hfill

%----------------------------------------------------------------------------------------

\end{minipage} % End of the sidebar mini page

%----------------------------------------------------------------------------------------

\end{document} 