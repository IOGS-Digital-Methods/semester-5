%%%%%%%%%%%%%%%%%%%%%%%%%%%%%%%%%%%%%%%%%%
% Engineering problems / LaTeX Template
%		Semester 5
%		Institut d'Optique Graduate School
%%%%%%%%%%%%%%%%%%%%%%%%%%%%%%%%%%%%%%%%%%
%	5N-ONIP-Block3	/ AM Modulation
%%%%%%%%%%%%%%%%%%%%%%%%%%%%%%%%%%%%%%%%%%
%
% Created by:
%	Julien VILLEMEJANE - 08/dec/2022
% Modified by:
%	
%
%%%%%%%%%%%%%%%%%%%%%%%%%%%%%%%%%%%%%%%%%%
% Professional Newsletter Template
% LaTeX Template
% Version 1.0 (09/03/14)
%
% Created by:
% Bob Kerstetter (https://www.tug.org/texshowcase/) and extensively modified by:
% Vel (vel@latextemplates.com)
% 
% This template has been downloaded from:
% http://www.LaTeXTemplates.com
%
% License:
% CC BY-NC-SA 3.0 (http://creativecommons.org/licenses/by-nc-sa/3.0/)
%
%%%%%%%%%%%%%%%%%%%%%%%%%%%%%%%%%%%%%%%%%

\documentclass[10pt]{article} % The default font size is 10pt; 11pt and 12pt are alternatives

\input{../../_latex_assets/5N_ONIP_structure.tex} % Include the document which specifies all packages and structural customizations for this template

\def\dirName{5N-ONIP-bloc3_sujet_AM}

\begin{document}


%----------------------------------------------------------------------------------------
%	DOCUMENT INFORMATIONS
%----------------------------------------------------------------------------------------
%----------------------------------------------------------------------------------------
%	DOCUMENT INFORMATIONS
%----------------------------------------------------------------------------------------
\def\module{Outils Numériques\\pour l'Ingénieur$\cdot$e\\en Physique}
\def\submodule{Outils Numériques}
\def\moduleSmall{5N-028-PHY / ONIP-1}
\def\year{2023-2024}
\def\problem{Bloc 3 - Modulation AM}
\def\problemName{Afficher et traiter des données provenant d'instruments de mesure}

\def\validation{50\%}

\def\scheduleCM{0}
\def\scheduleTD{0}
\def\scheduleTDcomputer{4}
\def\scheduleTP{0}

\def\workingTeam{Seul ou par binôme}

\def\workingSpecial{}

\def\keywords{Fichier CSV ; Graphique scientifique ; Transformée de Fourier ;
Modulation d'Amplitude ; Démodulation}


%----------------------------------------------------------------------------------------
%	HEADER IMAGE
%----------------------------------------------------------------------------------------

\begin{figure}[H]
\centering\includegraphics[width=0.5\linewidth]{../../_latex_assets/logo_iogs.png}
\end{figure}

%----------------------------------------------------------------------------------------
%	SIDEBAR - FIRST PAGE
%----------------------------------------------------------------------------------------

\begin{minipage}[t]{.35\linewidth} % Mini page taking up 30% of the actual page
\begin{mdframed}[style=sidebar,frametitle={\module}] % Sidebar box

%-----------------------------------------------------------
%	DOCUMENT DESCRIPTION
\begin{center}

\textit{\large \centering \year}
\end{center}


\centerline {\rule{.70\linewidth}{.25pt}} % Horizontal line

\begin{center}
	\textit{\large \moduleSmall}
\end{center}

\centerline {\rule{.70\linewidth}{.25pt}} % Horizontal line

\begin{center}
	\textbf{\problem} ( \validation )
\end{center}

\centerline {\rule{.70\linewidth}{.25pt}} % Horizontal line

%-----------------------------------------------------------

\textbf{Concepts étudiés}

\begin{itemize}
%----------------------------------------------------------------------------------------
%	COVERED CONCEPTS
%----------------------------------------------------------------------------------------
\item[\textsc{\scriptsize [Phys]}] Modulation d'amplitude
\item[\textsc{\scriptsize [Math]}] Transformée de Fourier
\item[\textsc{\scriptsize [Num]}] Signaux numériques
\item[\textsc{\scriptsize [Num]}] Figures scientifiques
\end{itemize}

\centerline {\rule{.70\linewidth}{.25pt}} % Horizontal line

%-----------------------------------------------------------

\textbf{Mots clefs}

\keywords

\centerline {\rule{.70\linewidth}{.25pt}} % Horizontal line

%-----------------------------------------------------------

\textbf{Sessions}

\begin{itemize}
\item[\textbf{\scheduleCM}] Cours(s) - 1h30
\item[\textbf{\scheduleTD}] TD(s) - 1h30
\item[\textbf{\scheduleTDcomputer}] TD(s) Machine - 2h00
\item[\textbf{\scheduleTP}] TP(s) - 4h30
\end{itemize}

\centerline {\rule{.70\linewidth}{.25pt}} % Horizontal line

{\large Travail}

\textbf{\workingTeam}

\textbf{\workingSpecial}


%-----------------------------------------------------------

\end{mdframed}


\centering
\begin{minipage}[t]{.95\linewidth}
\textbf{Institut d'Optique}\\
Graduate School, \textit{France}\\
\href{https://www.institutoptique.fr}{https://www.institutoptique.fr}

\medskip
\textbf{GitHub - Digital Methods}

\href{https://github.com/IOGS-Digital-Methods}{https://github.com/IOGS-Digital-Methods}

\end{minipage}

\end{minipage}\hfill % End the sidebar mini page 
%
%----------------------------------------------------------------------------------------
%	MAIN BODY - FIRST PAGE
%----------------------------------------------------------------------------------------
%
\begin{minipage}[t]{.60\linewidth} % Mini page taking up 66% of the actual page

\hypertarget{context}{\heading{\huge \problemName}{6pt}} % \hypertarget provides a label to reference using \hyperlink{label}{link text}

\centerline {\rule{.70\linewidth}{.25pt}} % Horizontal line

%% Short introduction 
%----------------------------------------------------------------------------------------
%	SHORT INTRODUCTION
%----------------------------------------------------------------------------------------
Les \textbf{expériences scientifiques}, les \textbf{essais industriels} sur des systèmes ou bien encore des \textbf{résultats de simulation} produisent énormément de \textbf{données}. 
Ces données sont souvent sauvegardées sous forme de \textbf{fichiers formatés} (format normalisé ou interne aux entreprises/laboratoires).

Il est alors indispensable de pouvoir \textbf{afficher les données} contenues dans ce type de fichier de manière claire et sans ambiguïté, avant d'en \textbf{extraire des informations pertinentes} par un traitement adapté.

\medskip

Vous traiterez dans cette séquence une \textbf{information modulée en amplitude}, acquise par un \textbf{oscilloscope numérique} et stockée dans un \textbf{fichier de type tableur}.

%%

\bigskip

%----------------------------------------------------------------------------------------
%	IN-TEXT BOX / Intended learning outcomes
%----------------------------------------------------------------------------------------

\begin{mdframed}[style=aavbox,frametitle={Acquis d'Apprentissage Visés}]

En résolvant ce problème, les étudiant$\cdot$e$\cdot$s seront capables de  :

\centerline {\rule{.40\linewidth}{.1pt}} % Horizontal line

\begin{center}
{\large \textsc{Côté Numérique}}
\end{center}

\begin{enumerate}
%----------------------------------------------------------------------------------------
%	Intended Learning Outcomes - Numerical Tools
%----------------------------------------------------------------------------------------
\item \textbf{Générer des signaux numériques} à partir de fonctions mathématiques 
\item \textbf{Définir et documenter des fonctions} pour générer des signaux numériques
\item \textbf{Produire des figures} claires et légendées à partir de signaux numériques - incluant un titre, des axes, des légendes
\item \textsc{\scriptsize [Bonus]} \textbf{Construire des bibliothèques de fonctions}
\end{enumerate}

\centerline {\rule{.40\linewidth}{.1pt}} % Horizontal line

\begin{center}
{\large \textsc{Côté Physique}}
\end{center}

\begin{enumerate}
%----------------------------------------------------------------------------------------
%	Intended Learning Outcomes - Physics
%----------------------------------------------------------------------------------------
\item \textbf{Analyser le contenu spectral} d'un signal électrique
\item \textbf{Déterminer les paramètres} d'une modulation d'amplitude
\item \textbf{Décoder} un signal modulé en amplitude
\end{enumerate}

\end{mdframed}
\medskip



%\begin{wrapfigure}[7]{l}[0pt]{0pt} % In-line figure with text wrapping around it
%\includegraphics[width=0.3\textwidth]{engPb_S5_01/placeholder.jpg}
%\end{wrapfigure}

\end{minipage} % End the main body - first page mini page

%----------------------------------------------------------------------------------------
%	MAIN BODY - SECOND PAGE
%----------------------------------------------------------------------------------------

\begin{minipage}[t]{.66\linewidth} % Mini page taking up 66% of the actual page

%----------------------------------------------------------------------------------------
%	IN-TEXT BOX / Deliverables
%----------------------------------------------------------------------------------------


\begin{mdframed}[style=intextbox,frametitle={Livrables attendus}] % Sidebar box

%----------------------------------------------------------------------------------------
%	DEVELIRABLES
%----------------------------------------------------------------------------------------

Pour valider cette session, vous devez \textbf{présenter} les \textbf{livrables suivants} lors de la séance 4 de ce bloc :

\begin{enumerate}
\item \textbf{Fonctions commentées} (selon la norme PEP 257) pour générer des signaux numériques appropriés
\item \textbf{Graphiques légendés} incluant toutes les données nécessaires à la bonne compréhension des données présentées : signal initial, transformée de Fourier du signal initial, signaux générés pour démoduler le signal, transformées de Fourier intermédiaires, signal démodulé
\item \textbf{Analyse des figures} en insistant sur la démarche ayant amené à la démodulation du signal
\item \textbf{BONUS : Fichiers démodulés} contenant les différents signaux démodulés
\end{enumerate}

\medskip

Ces livrables pourront prendre la forme d'un \textbf{compte-rendu} incluant une introduction à la problématique, les figures demandées ainsi que leur analyse.

Ce compte-rendu sera accompagné des \textbf{fichiers} \mbox{\textit{main.py}} et \mbox{\textit{signal\_processing.py}} contenant le programme principal permettant la génération des figures et de leurs légendes et les différentes fonctions commentées selon la norme PEP 257.

\textbf{Vous aurez 10 minutes lors de la séance 4 pour présenter l'ensemble de vos résultats et vos analyses.}


\end{mdframed}

%-----------------------------------------------------------

\hypertarget{stepbystep}{\heading{Données à traiter}{6pt}} % \hypertarget provides a label to reference using \hyperlink{label}{link text}

%----------------------------------------------------------------------------------------
%	STEP BY STEP
%----------------------------------------------------------------------------------------


Dans cette séquence, vous serez amenés à utiliser des données provenant d'un fichier de points issu d'un \textbf{oscilloscope}. Le fichier se nomme \mbox{\textsc{B3\_data\_01.csv}}.

Le signal qu'il contient est un enregistrement d'une \textbf{transmission d'informations modulées en amplitude} par un signal porteur sinusoïdal.

\medskip

Deux autres fichiers vous sont également proposés :
\begin{itemize}
	\item  \mbox{\textsc{B3\_data\_02.csv}} contenant un signal sonore modulé en amplitude à déchiffrer...
	\item  \mbox{\textsc{B3\_data\_03.csv}} contenant un ensemble de signaux modulés en amplitude à l'aide de différentes porteuses.
\end{itemize}




\centerline {\rule{.70\linewidth}{.25pt}} % Horizontal line

%-----------------------------------------------------------

\hypertarget{ressources}{\heading{Ressources}{6pt}} % \hypertarget provides a label to reference using \hyperlink{label}{link text}

%----------------------------------------------------------------------------------------
%	RESSOURCES
%----------------------------------------------------------------------------------------
Cette séquence est basée sur le langage Python.

Vous pouvez utiliser %l'environnement \textbf{JupyterHub@Paris-Saclay} - 
%\href{https://jupyterhub.ijclab.in2p3.fr/}{https://jupyterhub.ijclab.in2p3.fr/} ou 
l'environnement \textbf{Spyder 5} inclus dans \textit{Anaconda 3}.

Des tutoriels Python (et sur les bibliothèques classiques : Numpy, Matplotlib or Scipy) sont disponibles à l'adresse : \href{http://lense.institutoptique.fr/python/}{http://lense.institutoptique.fr/python/}. 



%----------------------------------------------------------------------------------------

\end{minipage}\hfill % End of the main body - second page mini page
\begin{minipage}[t]{.30\linewidth} % Mini page taking up 30% of the actual page

%----------------------------------------------------------------------------------------
%	SIDEBAR - SECOND PAGE
%----------------------------------------------------------------------------------------

\begin{mdframed}[style=sidebar,frametitle={}] % Sidebar box

\heading{Outils Numériques}{0pt}

\centerline {\rule{.40\linewidth}{.1pt}} % Horizontal line

\textbf{Fonctions et bibliothèques conseillées} :

%----------------------------------------------------------------------------------------
%	NUMERICAL TOOLS / BASICS
%----------------------------------------------------------------------------------------

\begin{itemize}
	\item \textbf{Numpy} gestion de matrices :
	\begin{itemize}
		\item \textbf{arange}
		\item \textbf{linspace}
		\item \textbf{logspace}	
	\end{itemize}
	\item \textbf{Matplotlib} affichage de données :
	\begin{itemize}
		\item \textbf{plotly} 
		\item \textbf{figure, plot}
		\item \textbf{subplot}
		\item \textbf{legend, title}	
		\item \textbf{xlabel, ylabel}	
		\item \textbf{show}	
	\end{itemize}	
	\item \textbf{Scipy} fonctions scientifiques :
	\begin{itemize}
		\item \textbf{fftpack} sublibrary
		\item \textbf{fft, ifft}
		\item \textbf{fftshift}		
		\item \textbf{fftfreq}
	\end{itemize}
\end{itemize}

\centerline {\rule{.40\linewidth}{.1pt}} % Horizontal line


\textbf{Outils avancés} :

%----------------------------------------------------------------------------------------
%	NUMERICAL TOOLS / ADVANCED
%----------------------------------------------------------------------------------------
\begin{itemize}
	\item \textbf{rcParams} de MatPlotLib.pyplot pour l'amélioration de l'affichage de courbes
\end{itemize}

\end{mdframed}\hfill

%----------------------------------------------------------------------------------------

\end{minipage} % End of the sidebar mini page

%----------------------------------------------------------------------------------------
\newpage

\hypertarget{stepbystep}{\heading{Etapes}{6pt}}

\begin{description}
	\item[Etape 1] \textbf{Lecture d'un fichier de points}
	
	\begin{itemize}
		\item \textit{Fichier : B3\_data\_01.csv} provenant d'un enregistrement sur un oscilloscope numérique.
		\item Lire un fichier texte / tableur
		\item Récupérer les données dans un vecteur
		\item Afficher les signaux contenus dans le fichier
	\end{itemize}
	

\qquad
	
	\item[Etape 2] \textbf{Calcul et affichage de la transformée de Fourier}
	\begin{itemize}
		\item Générer un signal périodique sinusoïdal
		\item Calculer la FFT d'un signal simple
		\item Afficher la FFT de ce signal
		\item Générer des signaux plus complexes et valider l'affichage de la FFT
	\end{itemize}
	
\qquad

	\item[Etape 3] \textbf{Simulation du phénomène de modulation/démodulation AM}
	\begin{itemize}
		\item Générer des signaux de tests
		\item Afficher la FFT de ces signaux
		\item Démoduler le signal et afficher la FFT
		\item Générer un fichier de points
	\end{itemize}

\qquad

	\item[Etape 4] \textbf{Démodulation d'un signal quelconque} 
	\begin{itemize}
		\item \textit{Fichier : B3\_data\_02.txt} provenant d'une génération d'un fichier en base 64 (fichier sonore initial en 24 kHz et 16 bits).
		\item Lire un fichier en base 64
		\item Récupérer les données dans un vecteur
		\item Afficher les signaux contenus dans le fichier
		\item Jouer le son décodé	
	\end{itemize}	
	
\qquad
	
	\item[Bonus 1] \textbf{Démodulation multi-porteuse}
	\begin{itemize}
		\item \textit{Fichier : B3\_data\_03.txt} provenant d'une génération d'un fichier en base 64 (fichier sonore initial en 160 kHz et 16 bits). Multi-porteuses sinusoïdales.
	\end{itemize}
	
	\item[Bonus 2] \textbf{Génération de fichiers modulés}
	
\end{description}

\newpage

\hypertarget{stepbystep}{\heading{Critères d'évaluation}{6pt}}

\begin{itemize}
	\item \textbf{METHODES NUMERIQUE}
	\begin{itemize}
		\item \textbf{Ecriture Matricielle / Vectorielle}
		\begin{itemize}
			\item utilisation des méthodes liées aux vecteurs/matrices (Numpy)
			\item aucune boucle \textbf{for} inutile
		\end{itemize}		 
		\item \textbf{Organisation en actions élémentaires}
		\begin{itemize}
			\item les étapes sont découpées en fonctionnalité plus simple à tester
		\end{itemize}
		\item \textbf{Description des tests de validation}
		\begin{itemize}
			\item chaque fonction a été testée
			\item chaque étape a été validée
		\end{itemize}
		\item \textbf{Organisation des informations à traiter}
		\begin{itemize}
			\item les données sont rangées dans des objets bien identifiés
		\end{itemize}
	\end{itemize}


	\item \textbf{PROGRAMMATION}
	\begin{itemize}
		\item \textbf{Ecriture globale du code et commentaires (PEP 8)}
		\begin{itemize}
			\item variables et fonctions respectant les conventions d'écriture standard
			\item commentaires utiles
		\end{itemize}		 
		\item \textbf{Utilisation, écriture de fonctions}
		\begin{itemize}
			\item paramètres et retours pertinents des fonctions
		\end{itemize}
		\item \textbf{Documentation des fonctions (PEP257)}
		\begin{itemize}
			\item paramètres et retours des fonctions sont documentés
		\end{itemize}
	\end{itemize}
	

	\item \textbf{INGENIEUR.E PHYSIQUE}
	\begin{itemize}
		\item \textbf{Graphiques pertinents et légendés}
		\begin{itemize}
			\item graphiques scientifiques (axes, titre...)
			\item axes des graphiques légendés (passage temps/fréquence)
		\end{itemize}		 
		\item \textbf{Organisation en actions élémentaires}
		\begin{itemize}
			\item les étapes sont découpées en fonctionnalité plus simple à tester
		\end{itemize}
		\item \textbf{Génération de données pertinentes de tests}
		\begin{itemize}
			\item données de test (amplitudes, fréquences...) pertinentes
		\end{itemize}
		\item \textbf{Analyse des données et validation modèle}
		\begin{itemize}
			\item comparaison avec la théorie
			\item analyse pertinente des signaux (temporels et fréquentiels)
		\end{itemize}
	\end{itemize}
	
	\item \textbf{AVANCEMENT}
	\begin{itemize}
		\item Etapes 1 et 2 : D
		\item Etapes 1, 2 et 3 : C
		\item Etapes 1, 2, 3 et 4 : B
		\item Bonus 1 : A
		\item Bonus 1 et 2 : A+
	\end{itemize}
\end{itemize}

	

%----------------------------------------------------------------------------------------

\end{document} 